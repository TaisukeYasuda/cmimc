\documentclass[10pt]{article}
\usepackage{amsmath, amssymb, amsthm}
\usepackage[margin=2cm]{geometry}
\usepackage[pdftex]{graphicx}
\usepackage{asymptote}
\usepackage{fancyhdr}
\pagestyle{fancy}
\rhead{}
\chead{\includegraphics[scale=0.12]{CMIMC-header.png}}
\lhead{}
\setlength{\headheight}{43pt}
\rfoot{}
\cfoot{}
\lfoot{}
\addtolength\footskip{-1cm}
\begin{document}\thispagestyle{empty}
\begin{center}

\vspace*{90pt}

\includegraphics[scale=0.23]{CMIMC-header.png}

\includegraphics[scale=0.3]{combo-header.png}

\vspace{1.6in}

\includegraphics[scale=0.20]{instruction-header.png}
\noindent\rule{17.7cm}{2pt}
\end{center}

\vspace{10pt}

\begin{enumerate}
\large
\item Do not look at the test before the proctor starts the round.

\item This test consists of 10 short-answer problems to be solved in 60 minutes.
	Each question is worth one point.

\item Write your name, team name, and team ID on your answer sheet. Circle the
	subject of the test you are currently taking.

\item Write your answers in the corresponding boxes on the answer sheets.

\item No computational aids other than pencil/pen are permitted.

\item All answers are integers.

\item If you believe that the test contains an error, submit your protest in writing to Porter 100.
\end{enumerate}

\newpage

\begin{center}
\huge\textbf{Combinatorics}\normalsize

\vspace{3pt}
\end{center}

\begin{enumerate}
\setlength{\itemsep}{5pt}

\item The phrase ``COLORFUL TARTAN'' is spelled out with wooden blocks, where blocks of the same letter are indistinguishable. How many ways are there to distribute the blocks among two bags of different color such that neither bag contains more than one of the same letter?

\item Six people each flip a fair coin. Everyone who flipped tails then flips their coin again. Given that the probability that all the coins are now heads can be expressed as simplified fraction $\tfrac{m}{n}$, compute $m+n$. %patrick

%\item Xinyu writes a computer program that randomly rearranges the digits $0,2,4,6$, and $8$ to create a five-digit number with no leading zeroes. If she executes this program once, the probability the program outputs an integer divisible by 4 can be written in the form $\tfrac{m}{n}$ where $m$ and $n$ are positive integers which share no common factors. What is $m+n$? %ID: 79; david

\item At CMU, markers come in two colors: blue and orange. Zachary fills a hat randomly with three markers such that each color is chosen with equal probability, then Chase shuffles an additional orange marker into the hat. If Zachary chooses one of the markers in the hat at random and it turns out to be orange, the probability that there is a second orange marker in the hat can be expressed as simplified fraction $\tfrac{m}{n}$. Find $m+n$. %ID: 66; patrick

\item Kevin colors three distinct squares in a $3\times 3$ grid red. Given that there exist two uncolored squares such that coloring one of them would create a horizontal or vertical red line, find the number of ways he could have colored the original three squares. %patrick

%\item Kevin is located on the top face of a regular dodecahedron. Every minute, he moves randomly to an adjacent face. Determine the expected amount of time, in minutes, that will pass before he moves to the bottom face for the first time. %ID: 48; patrick

\item Let $\mathcal{S}$ be a regular 18-gon, and for two vertices in $\mathcal{S}$ define the \textit{distance} between them to be the length of the shortest path along the edges of $\mathcal{S}$ between them (e.g. adjacent vertices have distance 1). Find the number of ways to choose three distinct vertices from $\mathcal{S}$ such that no two of them have distance 1, 8, or 9.

\item Shen, Ling, and Ru each place four slips of paper with their name on it into a bucket. They then play the following game: slips are removed one at a time, and whoever has all of their slips removed first wins. Shen cheats, however, and adds an extra slip of paper into the bucket, and will win when four of his are drawn. Given that the probability that Shen wins can be expressed as simplified fraction $\tfrac{m}{n}$, compute $m+n$. %ans: 184; victor

\item There are eight people, each with their own horse. The horses are arbitrarily arranged in a line from left to right, while the people are lined up in random order to the left of all the horses. One at a time, each person moves rightwards in an attempt to reach their horse. If they encounter a mounted horse on their way to their horse, the mounted horse shouts angrily at the person, who then scurries home immediately. Otherwise, they get to their horse safely and mount it. The expected number of people who have scurried home after all eight people have attempted to reach their horse can be expressed as simplified fraction $\tfrac{m}{n}$. Find $m+n$. %patrick

\item Brice is eating bowls of rice. He takes a random amount of time $t_1 \in (0,1)$ minutes to consume his first bowl, and every bowl thereafter takes $t_n = t_{n-1} + r_n$ minutes, where $t_{n-1}$ is the time it took him to eat his previous bowl and $r_n \in (0,1)$ is chosen uniformly and randomly. The probability that it takes Brice at least 12 minutes to eat 5 bowls of rice can be expressed as simplified fraction $\tfrac{m}{n}$. Compute $m+n$. %ID: 108; patrick

\item 1007 distinct potatoes are chosen independently and randomly from a box of 2016 potatoes numbered $1, 2, \dots, 2016$, with $p$ being the smallest chosen potato. Then, potatoes are drawn one at a time from the remaining 1009 until the first one with value $q < p$ is drawn. If no such $q$ exists, let $S = 1$. Otherwise, let $S = pq$. Then given that the expected value of $S$ can be expressed as simplified fraction $\tfrac{m}{n}$, find $m+n$. %ID: 48; patrick

%\item Let $\mathcal{G}$ denote the set of simple graphs.  

%\par For positive integers $m$ and $n$ with $m\geq2n$, let $C_{m,n}$ denote the graph with vertex sequence $\{v_i\}_{i=1}^m$ such that vertices $v_i$ and $v_j$ are adjacent iff either $|i-j|\leq n$ or $|i-j|\geq m-n$.  Determine the number of ordered pairs of positive integers $(m,n)$ with $1<2n\leq m\leq 100$ such that there exists some graph $H\in \mathcal{G}$ with $f(H)=C_{m,n}$.

%\par \textit{Note: }A graph is said to be \textit{simple} if it has no self-loops or multiple edges.  In other words, no edge connects a vertex to itself, and the number of edges connecting two distinct vertices is either $0$ or $1$. %ID: 89; david

\item For all positive integers $m\geq 1$, denote by $\mathcal{G}_m$ the set of simple graphs with exactly $m$ edges.  Find the number of pairs of integers $(m,n)$ with $1<2n\leq m\leq 100$ such that there exists a simple graph $G\in\mathcal{G}_m$ satisfying the following property: it is possible to label the edges of $G$ with labels $E_1$, $E_2$, $\ldots$, $E_m$ such that for all $i\neq j$, edges $E_i$ and $E_j$ are incident if and only if either $|i-j|\leq n$ or $|i-j|\geq m-n$.

\par \textit{Note: }A graph is said to be \textit{simple} if it has no self-loops or multiple edges. In other words, no edge connects a vertex to itself, and the number of edges connecting two distinct vertices is either $0$ or $1$.

\end{enumerate}
\end{document}
