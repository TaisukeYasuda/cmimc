\documentclass[10pt]{article}
\usepackage{amsmath, amssymb, amsthm}
\usepackage[top=2cm, left = 2cm, right = 2cm, bottom = 3cm]{geometry}
\usepackage[pdftex]{graphicx}
\usepackage{asymptote}
\usepackage{fancyhdr}
\pagestyle{fancy}
\rhead{}
\chead{\includegraphics[scale=0.12]{CMIMC-header.png}}
\lhead{}
\setlength{\headheight}{43pt}
\rfoot{}
\cfoot{}
\lfoot{}
\newcommand{\proposed}[1]
{
\vspace{5pt}
\noindent\textit{Proposed by #1}
}
\newcommand{\solution}
{
\vspace{5pt}
\noindent\textit{Solution.}\qquad
}
\newcounter{enum}
\setcounter{enum}{0}
\begin{document}

\begin{center}
\huge\textbf{Geometry Solutions Packet}\normalsize

\vspace{3pt}
\end{center}

\begin{enumerate}

\item Let $\triangle ABC$ be an equilateral triangle and $P$ a point on $\overline{BC}$.  If $PB=50$ and $PC=30$, compute $PA$.

\proposed{David Altizio}

\solution Let $M$ be the midpoint of $\overline{BC}$.  The fact that $PB=50$ and $PC=30$ implies that the side length of $\triangle ABC$ is $80$, so $AM=40\sqrt 3$.  Furthermore, it is easy to deduce that $BM=10$.  Therefore by the Pythagorean Theorem \[AP^2=AM^2+MB^2=(40\sqrt3)^2+10^2=4900\quad\implies\quad AP=\boxed{70}.\]

\item Let $ABCD$ be an isosceles trapezoid with $AD=BC=15$ such that the distance between its bases $AB$ and $CD$ is $7$.  Suppose further that the circles with diameters $\overline{AD}$ and $\overline{BC}$ are tangent to each other.  What is the area of the trapezoid?

\proposed{David Altizio}

\solution Let $T$ be the point of tangency of the two circles, and let $M$ and $N$ be the midpoints of $\overline{AD}$ and $\overline{BC}$ respectively.  Then $M$, $N$, and $T$ all lie on the same line, so \[MN=MT+TN=\frac12AD+\frac12BC = \frac12\cdot 15+\frac12\cdot 15 = 15.\] Now recall that the area of a trapezoid is $\tfrac12h(b_1+b_2)$, where $h$ is the distance between the bases of the trapezoid and $b_1$ and $b_2$ are said bases' lengths.  But recall that $MN$ is a midline of $ABCD$, meaning that its length is the average of the lengths of $AB$ and $CD$.  But this is precisely $\tfrac12(b_1+b_2)$!  Therefore the desired area is $7\cdot 15=\boxed{105}$.

\item Let $ABC$ be a triangle.  The angle bisector of $\angle B$ intersects $AC$ at point $P$, while the angle bisector of $\angle C$ intersects $AB$ at a point $Q$.  Suppose the area of $\triangle ABP$ is 27, the area of $\triangle ACQ$ is 32, and the area of $\triangle ABC$ is $72$.  The length of $\overline{BC}$ can be written in the form $m\sqrt n$ where $m$ and $n$ are positive integers with $n$ as small as possible.  What is $m+n$?

\proposed{David Altizio}

\solution For ease of typesetting let $[X]$ denote the area of region $X$.  Note that $[ABP]=27$ and $[ABC]=72$ implies that $[BCP]=72-27=45$, so by the Angle Bisector Theorem \[\dfrac{AB}{BC}=\dfrac{AP}{PC}=\dfrac{[ABP]}{[BPC]}=\dfrac{27}{45}=\frac{3}{5}.\] Through a similar process one may obtain $\tfrac{AC}{BC}=\tfrac45$.  Therefore $\triangle ABC$ is a 3-4-5 right triangle with a right angle at $A$.

\par Let $AB=3x$, $AC=4x$, and $BC=5x$ for some positive real $x$.  Then by the formula for area \[\dfrac12(3x)(4x) = 72\quad\implies\quad x=\sqrt{12}=2\sqrt3.\] Thus $BC=5x=10\sqrt3$ and the requested answer is $10+3=\boxed{13}$.

\setcounter{enum}{\theenumi}
\end{enumerate}

\hspace*{-\parindent}%
\begin{minipage}{0.85\textwidth}
\begin{enumerate}
\setcounter{enumi}{\theenum}
\item Andrew the Antelope canters along the surface of a regular icosahedron, which has twenty equilateral triangle faces and edge length 4. (A three-dimensional image of an icosahedron is shown to the right.) If he wants to move from one vertex to the opposite vertex, the minimum distance he must travel can be expressed as $\sqrt{n}$ for some integer $n$.  Compute $n$. %Patrick Lin
\setcounter{enum}{\theenumi}
\end{enumerate}

\end{minipage} \hfill
\begin{minipage}{0.12\textwidth}

\includegraphics[scale=0.20]{icosahedron.jpg}
\end{minipage}

\hfill
\begin{minipage}{0.95\textwidth}
\proposed{Patrick Lin}

\solution Looking at the icosahedral net, it is clear that the desired length is equal to the hypotenuse of a right triangle with one leg equal to the height of a triangular face and the other leg equal to $\tfrac52$ of the side length of a face. Hence Pythagorean theorem yields $10^2+(2\sqrt{3})^2 = \boxed{112}$.
\end{minipage}

\begin{enumerate}
\setcounter{enumi}{\theenum}

\item Let $\mathcal{P}$ be a parallelepiped with side lengths $x$, $y$, and $z$.  Suppose that the four space diagonals of $\mathcal{P}$ have lengths $15$, $17$, $21$, and $23$.  Compute $x^2+y^2+z^2$.

\proposed{David Altizio and Joshua Siktar}

\solution Recall the Parallelogram Law in two dimensions, which states that if $x$ and $y$ are elements of $\mathbb{R}^2$ then $|x+y|^2+|x-y|^2=2(|x|^2+|y|^2)$.  (This is true by the Law of Cosines.)  I claim that this can be extended further.  Indeed, for any three dimensional vectors $x$, $y$, and $z$ in $\mathbb{R}^3$, the identity \[4(|x|^2+|y|^2+|z|^2)=|x+y+z|^2+|x+y-z|^2+|x-y+z|^2+| -x+y+z|^2\] is true.  To prove this, we use the two-dimensional version repeatedly.  Note that $0$, $x+y$, $z$, and $x+y+z$ form a parallelogram, which means that \[2(|x+y|^2+|z|^2)=|x+y+z|^2+|x+y-z|^2.\] Similarly, since $0$, $x-y$, $z$, and $x-y+z$ form a parallelogram, we have \[2(|x-y|^2+|z|^2)=|x-y+z|^2+| -x+y+z|^2.\] Adding these together yields \[2(|x+y|^2+|x-y|^2)+4|z|^2=|x+y+z|^2+|x+y-z|^2+|x-y+z|^2+| -x+y+z|^2\] and using the parallelogram law on the LHS one last time yields the desired equality.

\par Returning back to the original problem, we have \[4(x^2+y^2+z^2)=15^2+17^2+21^2+23^2=2(16^2+1)+2(22^2+1)=2(16^2+22^2) + 4,\] which means that $x^2+y^2+z^2=2(8^2+11^2)+1=\boxed{371}$.

\par\textit{Remark.} This version of the parallelogram law can be extended to hold true in all dimensions.  Formally, \[2^n\sum_{i=1}^n|z_i|^2=\sum_{(e_1,\ldots, e_n)\in\{+1,-1\}^n}\left|\sum_{i=1}^ne_iz_i\right|^2\] for vectors $z_1,\ldots,z_n\in\mathbb{R}^n$.

\item In parallelogram $ABCD$, angles $B$ and $D$ are acute while angles $A$ and $C$ are obtuse.  The perpendicular from $C$ to $AB$ and the perpendicular from $A$ to $BC$ intersect at a point $P$ inside the parallelogram.  If $PB=700$ while $PD=821$, what is $AC$?

\proposed{David Altizio}

\solution First note that $P$ is the orthocenter of $\triangle ABC$.  Furthermore, note that from the perpendicularity $DA\perp AP$ and $DC\perp CP$, so quadrilateral $DAPC$ is cyclic.  Furthermore, $DP$ is a diameter of circle $(DAPC)$.  This is the circumcircle of $\triangle DAC$, which is congruent to $BCA$.  As a result, if $R$ is the circumradius of $\triangle ABC$, then $PD=2R$.

\begin{figure}[ht]
	\centering
	\begin{asy}
import olympiad;
size(250);
defaultpen(linewidth(0.8));
pair A = dir(70), B = origin, C = (1.5, 0), D = A + C, P = orthocenter(A,B,C), X = foot(A,B,C), Y = foot(C,B,A);
draw(A--B--C--D--A--X^^C--Y^^B--P--D);
draw(circumcircle(A,D,C),linetype("4 4"));
draw(rightanglemark(A,X,C,2)^^rightanglemark(B,Y,C,2));
label("$A$",A,NW);
label("$B$",B,SW);
label("$C$",C,SE);
label("$D$",D,NE);
label("$P$",P,2*dir(55));
\end{asy}
\end{figure}

\par Now I claim that $PB=2R\cos B$.  To prove this, reflect $P$ across $AB$ to point $P'$.  It is well-known that $P'$ lies on the circumcircle of $\triangle ABC$, so in particular the circumradii of $\triangle APB$ and $\triangle ACB$ are equal.  But then by Law of Sines \[\dfrac{BP}{\sin\angle BAP}=\dfrac{BP}{\cos B}=2R\quad\implies\quad BP = 2R\cos B\] as desired.  (An alternate way to see this is through the diagram itself: from right triangle trigonometry on triangles $DAP$ and $DCP$ it is not hard to see that $PA=2R\cos A$ and $PC=2R\cos C$, which by symmetry suggests $PB=2R\cos B$.)

\par Finally, note that by Law of Sines again we have $AC=2R\sin B$, so \[AC^2+BP^2=(2R\sin B)^2 + (2R\cos B)^2 = (2R)^2(\sin^2 B+\cos^2 B) = PD^2.\] Hence \[AC^2=PD^2-PB^2=821^2-700^2=(821-700)(821+700)=11^2\cdot 39^2\] and so $AC=11\cdot 39=\boxed{429}$.

\item Let $ABC$ be a triangle with incenter $I$ and incircle $\omega$.  It is given that there exist points $X$ and $Y$ on the circumference of $\omega$ such that $\angle BXC=\angle BYC=90^\circ$.  Suppose further that $X$, $I$, and $Y$ are collinear.  If $AB=80$ and $AC=97$, compute the length of $BC$.

\proposed{David Altizio}

\solution Let $\Omega$ be the circle with diameter $\overline{AC}$.  Then $X$ and $Y$ are the intersection points of $\omega$ and $\Omega$, so $XY$ is the radical axis of $\omega$ and $\Omega$.  The condition that $X$, $I$, and $Y$ are collinear implies that $I$ lies on the radical axis of these two circles.

\par Let $M$ be the midpoint of $\overline{BC}$ and $D$ the point of tangency of $\omega$ with $BC$.  The power of $I$ with respect to $\omega$ is $r^2$, while the power of $I$ with respect to $\Omega$ is \[MB^2-MI^2=\left(\frac a2\right)^2 - (ID^2+DM^2) = \frac{a^2}4-\left(r^2+\left(\frac a2 - (s-b)\right)^2\right)=a(s-b)-r^2-(s-b)^2.\] Setting these equal to each other yields \[2r^2 = a(s-b)-(s-b)^2 = (s-b)(a+b-s)=(s-b)(s-c).\] Now recall that Heron's Formula states $(rs)^2 = K^2 = s(s-a)(s-b)(s-c)$.  Plugging in our equality from above and cancelling like mad leads to \[s=2(s-a)\quad\implies\quad a+b+c=2(b+c-a)\quad\implies\quad b+c=3a.\] Hence $BC=\tfrac{AB+AC}3=\tfrac{80+97}3=\boxed{59}$.

\item Suppose $ABCD$ is a convex quadrilateral satisfying $AB=BC$, $AC=BD$, $\angle ABD = 80^\circ$, and $\angle CBD = 20^\circ$.  What is $\angle BCD$ in degrees?

\proposed{David Altizio}

\solution Construct a point $X$ outside $\triangle ABC$ such that $\triangle BCD\cong\triangle AXC$.  (This can be done from the fact that $AC=BD$.)  Then from $AB=BC$ we know $\angle BAC=40^\circ$, so $\angle BAX = 40^\circ+20^\circ = 60^\circ$.  Combining this with $AX=BC=BA$ gives that $\triangle ABX$ is equilateral.

\begin{figure}[h]
	\centering
	\begin{asy}
	import olympiad;
	size(120);
	defaultpen(linewidth(0.8));
	pair A = dir(0), B = origin, C = dir(100), D = abs(A-C) * dir(80), X = dir(60);
	draw(A--B--C--cycle^^B--D--C);
	draw(A--X--C,linetype("4 4"));
	label("$A$",A, SE);
	label("$B$",B, SW);
	label("$C$",C, W);
	label("$D$",D, N);
	label("$X$",X, NE);
\end{asy}
\end{figure}

\par From here, note that $\angle CBX = \angle CBA - 60^\circ=40^\circ$, and since $\triangle CBX$ is isosceles $\angle BXC=70^\circ$.  Thus \[\angle BCD = \angle AXC = \angle BXC + \angle AXB = 70^\circ+60^\circ=\boxed{130^\circ}.\]

\item Let $\triangle ABC$ be a triangle with $AB=65$, $BC=70$, and $CA=75$.  A semicircle $\Gamma$ with diameter $\overline{BC}$ is erected outside the triangle.  Suppose there exists a circle $\omega$ tangent to $AB$ and $AC$ and furthermore internally tangent to $\Gamma$ at a point $X$.  The length $AX$ can be written in the form $m\sqrt{n}$ where $m$ and $n$ are positive integers with $n$ not divisible by the square of any prime.  Find $m+n$.

\proposed{David Altizio}

\solution Scale down by a factor of $5$, so that $AB=13$, $BC=14$, and $CA=15$.  Let $\kappa$ denote the incircle of $\triangle ABC$.  The key is to recognize that by Monge's Theorem (or simply composite homotheties) $AX$ passes through the exsimilicenter $P$ of $\kappa$ and $\Gamma$.  Since both of these circles are fixed, $P$ is also fixed.  Thus it suffices to determine the location of $P$ and use this to find the location of $X$.

\begin{figure}[h]
	\centering
	\begin{asy}
import olympiad;
size(230);
defaultpen(linewidth(0.8));
pair A = (25,60), B = origin, C = (70,0), D = (25-15/2,0), Dn = 2 * D - A, I = incenter(A,B,C), P = (35*2/3, 140/3);
path circ = arc(C/2,35,180,360);
path circ2 = arc(C/2,35,0,180);
pair X = intersectionpoint(circ, A--Dn);
draw(A--B--C--A--X^^circ);
pair Iw = extension(X, C/2, A, bisectorpoint(B,A,C));
draw(circle(Iw,abs(Iw-X)),linetype("4 4"));
pair M = C/2;
path tang = circle((P+M)/2, abs((P-M)/2));
pair[] T = intersectionpoints(tang, circ2);
draw((2*T[0]-P)--P--(2*T[1]-P),linetype("2 2"));
dot(X^^M^^I^^P);
draw(incircle(A,B,C),linetype("4 4"));
draw(circ2, linetype("4 4"));
label("$A$",A,N);
label("$B$",B,W);
label("$C$",C,E);
label("$X$",X,dir(C/2--X));
label("$M$",M,S);
label("$I$",I,E);
label("$P$",P,NE);
\end{asy}
\end{figure}

Denote by $I$ the incenter of $\triangle ABC$ and by $M$ the midpoint of $\overline{BC}$.  Furthermore, let $I_0$ and $P_0$ be the feet of the perpendiculars from $I$ and $P$ respectively to $BC$.  We can easily compute that the radii of $\kappa$ and $\Gamma$ are $4$ and $7$ respectively, so by the definition of exsimilicenter, $\tfrac{PI}{PM}=\tfrac47$.  This in turn implies $\tfrac{P_0I_0}{P_0M}=\tfrac47$.  It is readily seen that $I_0M=1$, so therefore $P_0M=\tfrac73$.  Similar reasoning yields $P_0P=\tfrac{28}3$.

\par Briefly turn to coordinates to make conceptualization easier.  Set up a coordinate system where $M$ is the origin and $BC$ is the $x$-axis.  Then $P$ has coordinates $(-\tfrac73,\tfrac{28}3)$ and $A$ has coordinates $(-2,12)$, so the slope of line $AP$ is $8$.

\par Revert back to the Euclidean Plane.  Let $D=AX\cap BC$, and let $X_0$ be the foot of the perpendicular from $X$ to $BC$.  Set $X_0D$ to be $t$.  Then $XX_0=8t$.  Furthermore, $DM$ can be easily computed to be $\tfrac32+2=\tfrac72$ by the definition of slope.  Thus by Pythagorean Theorem on $\triangle MX_0X$, \[\left(t+\dfrac72\right)^2+(8t)^2=7^2\quad\implies\quad t=\frac7{10}.\] A few applications of the Pythagorean Theorem yield $AX=\tfrac{11\sqrt{65}}5$.  Scaling back up by a factor of $5$ yields $m=11$, $n=65$, and $m+n=\boxed{076}$.

\item Let $\triangle ABC$ be a triangle with circumcircle $\Omega$ and let $N$ be the midpoint of the major arc $\widehat{BC}$.  The incircle $\omega$ of $\triangle ABC$ is tangent to $AC$ and $AB$ at points $E$ and $F$ respectively.  Suppose point $X$ is placed on the same side of $EF$ as $A$ such that $\triangle XEF\sim\triangle ABC$.  Let $NX$ intersect $BC$ at a point $P$.  If $AB=15$, $BC=16$, and $CA=17$, then compute $\tfrac{PX}{XN}$.

\proposed{David Altizio}

\solution We solve for general $a$, $b$, and $c$.  We start off by proceeding through a series of lemmas.

\begin{figure}[h]
	\centering
	\begin{asy}
import olympiad;
size(400);
defaultpen(linewidth(0.8)+fontsize(11pt));
pair C = dir(-40), A = dir(125), B = dir(220), N = dir(90);
pair I = incenter(A,B,C), D = foot(I, B, C), E = foot(I, A, C), F = foot(I, A, B);
path circ1 = circumcircle(A,B,C), circ2 = circumcircle(A,E,F);
pair pt1 = 3 * I - 2 * D, pt2 = 4 * I - 3 * D, X = intersectionpoint(pt1--pt2, circ2);
pair inter[] = intersectionpoints(circ1, circ2);
pair Q = inter[1], P = extension(N,Q,B,C);
dot(incenter(A,B,C)^^N^^D);
draw(A--B--C--A^^circ1^^X--F--E--X^^incircle(A,B,C)^^B--P--N);
draw(circumcircle(A,E,F),linetype("4 4"));
draw(A--X--D^^P--E,linetype("3 3"));
label("$A$",A,dir(incenter(A,B,C)--A));
label("$B$",B,dir(origin--B));
label("$C$",C,dir(origin--C));
label("$D$",D,S);
label("$E$",E,dir(10));
label("$F$",F,2 * dir(230));
label("$X$",X,dir(90));
label("$I$",incenter(A,B,C),SE);
label("$Q$",Q,dir(origin--Q));
label("$P$",P,dir(F--P));
label("$N$",N,dir(90));
\end{asy}
\end{figure}

\par\textbf{LEMMA 1: }$AX\parallel BC$.

\begin{proof}Let $I$ be the incenter of $\triangle ABC$.  Note that since $\angle EXF=\angle EAF$, $X$ lies on the circumcircle of $\triangle AEF$.  Now remark that since \[\angle FID+\angle FIX=180^\circ-\angle B + \angle XEF=180^\circ-\angle B + \angle B = 180^\circ,\] we have $D$, $I$ and $X$ collinear, i.e. $XI\perp BC$.  Furthermore, $A$ and $I$ are antipodal with respect to $(AEF)$, so $\angle AXI=90^\circ$.  Hence $AX\parallel BC$ as desired.\end{proof}

\par\textbf{LEMMA 2: }Denote by  $Q$ the second intersection point of $\Omega$ with $(AEF)$.  Then $Q$ lies on $\overline{PN}$.

\begin{proof} Extend $AX$ to hit $\Omega$ again at $A'$.  Then $AA'CB$ is an isosceles trapezoid.  Furthermore, $N$ is the midpoint of $\widehat{AA'}$, so $\angle AQN=\angle A'QN$.

\par Now consider the spiral similarity sending $\triangle XEF$ to $\triangle A'BC$.  This spiral similarity is centered at $Q$ (a well-known fact - provable by angle chasing).  Since this spiral similarity sends $A$ to $N$ (both are midpoints of their respective arcs), we have $\triangle QAX\sim\triangle QNA'$, i.e. $\angle AQX=\angle NQA'$.  Hence $N$, $X$, and $Q$ are collinear, leading to the desired.
\end{proof}

\par\textbf{LEMMA 3:} $EF$ passes through $P$.

\begin{proof}Note that by simple angle chasing \[\angle XQF=180^\circ-\angle XEF=180^\circ-\angle ABC=\angle PBF.\] This implies that quadrilateral $PQFB$ is cyclic, so $\angle PBQ=\angle PFQ$.  But since $Q$ is the center of spiral similarity sending $EF$ to $BC$, we also have $\angle QFE=\angle QBC$.  Hence since $P$, $B$, and $C$ are collinear we must also have $P$, $E$, and $F$ collinear.  
\end{proof}

Now we compute.  Remark that by Power of a Point $PQ\cdot PX=PF\cdot PE=PD^2$ and $PQ\cdot PN=PB\cdot PC$, so \[\dfrac{XN}{PX}=\dfrac{PN}{PX}-1=\dfrac{PN\cdot PQ}{PX\cdot PQ} - 1 = \dfrac{PB\cdot PC}{PD^2}-1.\] To compute $PB$, remark that by either Menelaus or harmonic divisions we may obtain $\tfrac{PB}{PC}=\tfrac{DB}{DC}$.  Since $BD=s-b$ and $CD=s-c$, it is easy to find that $PB=\tfrac{a(s-b)}{b-c}$.  This means that $PC=\tfrac{a(s-c)}{b-c}$ and \[PD=\dfrac{a(s-b)}{b-c}+(s-b)=(s-b)\left(\dfrac{a}{b-c}+1\right)=(s-b)\left(\dfrac{a+b-c}{b-c}\right)=\dfrac{2(s-c)(s-b)}{b-c}.\] As a result, \[\dfrac{PB\cdot PC}{PD^2}=\dfrac{a^2(s-b)(s-c)/(b-c)^2}{(2(s-b)(s-c)/(b-c))^2}=\frac{a^2}{4(s-b)(s-c)}.\] Hence \[\dfrac{XN}{PX}=\dfrac{a^2}{4(s-b)(s-c)}-1=\dfrac{16^2}{4(24-15)(24-17)}-1=\dfrac{8^2}{9\cdot 7}-1=\dfrac{1}{63},\] so $\tfrac{PX}{XN}=\boxed{63}$.

\end{enumerate}

\end{document}