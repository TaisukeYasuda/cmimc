\documentclass[10pt]{article}
\usepackage{amsmath, amssymb, amsthm, mathtools, enumerate}
\usepackage[top=2cm, left = 2cm, right = 2cm, bottom = 3cm]{geometry}
\usepackage[pdftex]{graphicx}
\usepackage{asymptote}
\usepackage{fancyhdr}
\newcommand{\N}{\mathbb{N}}
\newcounter{enum}
\setcounter{enum}{0}
\pagestyle{fancy}
\rhead{}
\chead{\includegraphics[scale=0.17]{CMIMC-header-2017.png}}
\lhead{}
\setlength{\headheight}{43pt}
\rfoot{}
\cfoot{}
\lfoot{}
\newcommand{\proposed}[1]
{
\vspace{5pt}
\noindent\textit{Proposed by #1}
}
\newcommand{\solution}
{
\vspace{5pt}
\noindent\textit{Solution.}\qquad
}
\DeclarePairedDelimiter\abs{\lvert}{\rvert}
\begin{document}

\begin{center}
\huge\textbf{Combinatorics Solutions Packet}\normalsize

\vspace{3pt}
\end{center}

\begin{enumerate}

\item Robert colors each square in an empty 3 by 3 grid either red or green. Find the number of colorings such that no row or column contains more than one green square.

\proposed{Patrick Lin}

\solution Let the grid have $k$ green squares. Then $0 \le k \le 3$, otherwise by Pigeonhole some row must contain two green squares. Note also that order does not matter. $k = 0$ gives one solution and $k = 1$ gives nine solutions, clearly. $k = 2$ gives $\frac12(9\cdot4) = 18$ since there are only four choices remaining after coloring the first square green, and $k = 3$ similarly gives $\frac16(9\cdot4\cdot1) = 6$ solutions. This gives a total of $1+9+18+6 = \boxed{34}$ colorings.





\item Let $S$ be a subset of $\{1,2,\dots,2017\}$ such that for any two distinct elements in $S$, both their sum and product are not divisible by seven. Compute the maximum number of elements that can be in $S$.

\proposed{Patrick Lin}

\solution Note that $7 \mid 2016$, and so there 288 numbers that are $i \pmod 7$ for $i \neq 1$, and 289 for $i = 1$. From the product condition, it follows that $S$ cannot contain anything divisible by seven. From the sum condition, if $x \in S$ and $x \equiv 1 \pmod 7$, then nothing that is $6 \pmod 7$ is allowed, and similarly for the other pairs; hence, from each of the pairs $(1,6),(2,5),(3,4)$ we can only choose one residue class. We can take everything of a residue class, and so the maximum is $289 + 288 + 288 = \boxed{865}$.






\item Annie stands at one vertex of a regular hexagon. Every second, she moves independently to one of the two vertices adjacent to her, each with equal probability. Determine the probability that she is at her starting position after ten seconds.

\proposed{Phillip Wang}

\solution Call a clockwise move L and a counter-clockwise move R. Then she is at her starting position after ten seconds if the moves have 5 L's and 5 R's, 8 L's and 2 R's, or 2 L's and 8 R's, which can occur in $\binom{10}{5} + 2\cdot\binom{10}{8} = 342$ ways. The answer is hence $\frac{342}{2^{10}} = \boxed{\tfrac{171}{512}}$.






\item At a certain pizzeria, there are five different toppings available and a pizza can be ordered with any (possibly empty) subset of them on it. In how many ways can one order an unordered pair of pizzas such that at most one topping appears on both pizzas and at least one topping appears on neither?

\proposed{Patrick Lin}

\solution If no topping appears on both pizzas, then there are $3^5 - 2^5 = 211$ ordered pairs, since each topping is either on the first, second, or neither pizza and at least one must be on neither. If one topping appears on both pizzas, then there are five ways to choose that one and $3^4 - 2^4 = 65$ ways to assign the rest, for 325 ordered pairs total.

\par To account for ordering, note that the only way to order two identical pizzas if they have the same singular topping or have no toppings; these contribute 6 pairs. Hence the total number of unordered pairs is $\frac{211+325+6}{2} = \boxed{271}$.





\newpage
\item Emily draws six dots on a piece of paper such that no three lie on a straight line, then draws a line segment connecting each pair of dots. She then colors five of these segments red. Her coloring is said to be \emph{red-triangle-free} if for every set of three points from her six drawn points there exists an uncolored segment connecting two of the three points. In how many ways can Emily color her drawing such that it is red-triangle-free?

\proposed{David Altizio}

\solution We instead count the complement, i.e. the number of colorings that do contain a red triangle.

\par We first show that Emily's coloring can result in at most two red triangles.  Note that any two triangles can share at most one side - otherwise, the two triangles would be identical.  Hence if $N$ is the number of triangles in the coloring, then the number of edges $E$ must satisfy \[E\geq 3N - \binom{N}2 = 3N - \dfrac{N(N-1)}2 = \dfrac{N(7-N)}2.\] This is at most $5$ when $N\leq 2$ or $N\geq 5$, but clearly $N\geq 5$ is absurd, so indeed $N\leq 2$ as desired.  It is also worth noting that at $N=2$ equality holds, so in that case exactly one side must be common to both triangles.

\par We now use the Principle of Inclusion-Exclusion to find the number of colorings with at least one red triangle.  Choose one of the $\binom{6}{3} = 20$ possible right triangles.  There are $\binom{6}2 - 3 = 12$ edges left to choose from, and we can choose them in $\binom{12}2 = 66$ ways.  Thus, the number of colorings with at least one red triangle is $66\cdot 20 = 1320$ using this analysis.   However, this overcounts the number of colorings that contain two red triangles.  To tackle this case, recall that in our combinatorial analysis above we determined that such a case holds when exactly one side is common to both triangles.  There are $15$ ways to choose this side.  From here, note that the coloring is determined by choosing two of the other four dots and connecting each endpoint of the chosen side to each of the two chosen dots.  There are $6$ ways to choose the dots, and so there are $15\cdot 6 = 90$ colorings with two red triangles.  Thus the true number of colorings with at least one red triangle is $1320 - 90 = 1230$.

\par Since the total number of possible colorings is \[\binom{15}5 = \dfrac{15\cdot 14\cdot 13\cdot 12\cdot 11}{5\cdot 4\cdot 3\cdot 2\cdot 1} = 7\cdot 13\cdot 3 \cdot 11 = 3003,\] the requested answer is $3003 - 1230 = \boxed{1773}$.





\item Boris plays a game in which he rolls two standard four-sided dice independently and at random, and at the end of the game receives a number of dollars equal to the product of the two rolled numbers. After the initial roll of both dice, however, he can pay two dollars to reroll one die of choice, and he is allowed to pay to reroll as many times as he wishes. If Boris plays to maximize his expected gain, how much money, in dollars, can he expect to win by playing once?

\proposed{Patrick Lin}

\solution For $1 \le i,j \le 4$, let $E_{ij}$ be the expected value of this game given the initial roll was $(i,j)$. Note that $E_{ij} = E_{ji}$, so assume for now that $i \le j$. Since it is clear that if a reroll should be used it should be used to reroll the smaller number, we have
\[E_{ij} = \max\left(ij, -2 + \frac14 \sum_{1\le k\le 4} E_{kj}\right),\]
where it is the left expression when we do not reroll and the right one if we do.
Begin by noting that $E_{44} = 16$, since we can do no better. $E_{34} = 12$, since rolling until we obtain a pair of fours loses six dollars in expectation. Similarly, $E_{33} = 9$, since in expectation we lose eight dollars for a gain of $E_{34} - 9 = 3$. We should reroll on pairs $(1,4)$ and $(2,4)$ since even rerolling exactly once does no worse and being able to reroll multiple times is better, which gives $E_{14} = E_{24} = 10$.

\par Similarly, we find that rerolling on pairs $(1,3)$ and $(2,3)$ is optimal (clearly we should reroll on the former, and either through intuition or trying out the two strategies gives us the latter). Finally, we can also determine that on pairs $(1,1),(1,2),(2,2)$ we should reroll, allowing us to compute
\[E_{ij} = 
\begin{cases}
\frac{17}{4} & (i,j) = (1,1),(1,2),(2,2) \\
\frac{13}{2} & (i,j) = (1,3),(2,3) \\
9 & (i,j) = (3,3) \\
10 & (i,j) = (1,4),(2,4) \\
12 & (i,j) = (3,4) \\
16 & (i,j) = (4,4).
\end{cases}\]
The expected value of the game is hence
\[\frac{1}{16} \sum_{1 \le i,j \le 4} E_{ij} = \boxed{\frac{33}{4}}.\]




\item Given a finite set $S \subset \mathbb{R}^3$, define $f(S)$ to be the mininum integer $k$ such that there exist $k$ planes that divide $\mathbb{R}^3$ into a set of regions, where no region contains more than one point in $S$. Suppose that
\[M(n) = \max\{f(S) : \abs{S} = n\} \text{ and } m(n) = \min\{f(S) : \abs{S} = n\}.\]
Evaluate $M(200) \cdot m(200)$.

\proposed{Patrick Lin}

\solution First, it is clear that $M(200) = 199$; every plane adds one region at minimum, and equality is achieved when $S$ consists of 200 collinear points.

\par Now, note that $m(n)$ is equal to the minimum integer $k$ such that the maximum number of regions that $k$ planes can divide $\mathbb{R}^3$ in is at least $n$. We claim that $k$ planes can divide the space into at most $\binom{k}{3} + \binom{k}{2} + \binom{k}{1} + \binom{k}{0}$ regions; this follows from the fact that every region can be associated with a unique subset of the $k$ planes of size at most three via the planes that bound the region from below (this, in turn, comes from the fact that three non-parallel planes intersect at a point). Noting that 
\[\binom{10}{3} + \binom{10}{2} + \binom{10}{1} + \binom{10}{0} = 176 < 200 < 232 = \binom{11}{3} + \binom{11}{2} + \binom{11}{1} + \binom{11}{0},\]
it then follows that $m(200) = 11$, and so the answer is $199\cdot11 = \boxed{2189}$.





\item Andrew generates a finite random sequence $\{a_n\}$ of distinct integers according to the following criteria:
\begin{itemize}
\item $a_0 = 1$, $0 < \abs{a_n} < 7$ for all $n$, and $a_i \neq a_j$ for all $i < j$.
\item $a_{n+1}$ is selected uniformly at random from the set $\{a_n - 1, a_n + 1, -a_n\}$, conditioned on the above rule. The sequence terminates if no element of the set satisfies the first condition.
\end{itemize}
For example, if $(a_0, a_1) = (1, 2)$, then $a_2$ would be chosen from the set $\{-2,3\}$, each with probability $\tfrac12$. Determine the probability that there exists an integer $k$ such that $a_k = 6$.

\proposed{Patrick Lin}

\solution Equivalently, consider a random walk on a 2-by-6 grid of squares, where we begin at the upper left corner. We then wish to find the probability that we reach the upper right corner; note, however, that this is equal to the probability that we ever make it to the last column, for if we reach the lower right corner first the next move must be to the upper right corner.

\par Define $A_n$ to be the probability that we reach the last column of a 2-by-$n$ grid, and $B_n$ to be the same except where it is possible to move left via moving to the other row then left, as depicted below. Then we wish to find $A_6$, and note that moving to the left at any point guarantees that we can no longer make it to the last column.

\begin{center}
\includegraphics[scale=0.3]{Combo2017-2}
\end{center}

\par Using the diagram above, we find the recurrence relations
\begin{gather*}
A_n = \frac12 A_{n-1} + \frac12 B_{n-1} \\
B_n = \frac14 A_{n-1} + \frac12 B_{n-1},
\end{gather*}
where $A_1 = B_1 = 1$. The relation for $A_n$ is obvious, and for $B_n$ we get the $B_{n-1}$ term when moving right, and the $A_{n-1}$ term when moving along the column, accounting for the probability that we move left and can no longer reach the last column without duplicating a square. Either through direction computation, rearranging the equations to get the recurrence
\[A_n - A_{n-1} = -\frac18 A_{n-2},\]
or by solving the characteristic to get
\[A_n = \frac{\sqrt{2}}{4^n}\left((2+\sqrt{2})^n - (2-\sqrt{2})^n\right),\]
we obtain $A_6 = \boxed{\tfrac{35}{64}}$.





\item At a conference, six people place their name badges in a hat, which is shaken up; one badge is then distributed to each person such that each distribution is equally likely. Each turn, every person who does not yet have their own badge finds the person whose badge they have and takes that person's badge. For example, if Alice has Bob's badge and Bob has Charlie's badge, Alice would have Charlie's badge after a turn. Compute the probability that everyone will eventually end up with their own badge.

\proposed{Patrick Lin}

\solution We treat each assignment of badges as a permutation, which we can then decompose uniquely into cycles. Note that if one member of a cycle obtains their own badge after some number of turns, then every other member in the same cycle also has their own badge. For each cycle $\tau$, one turn is the equivalent of applying the cycle, and so in one turn it becomes $\tau^2$, which becomes $\tau^4$ on the next turn, and so on.

\par Hence, this process will terminate if and only if the cycle decomposition consists only of cycles whose lengths are powers of two. This gives rise to the recurrence
\[f(n) = f(n-1) + (n-1)f(n-2) + \frac{(n-1)!}{(n-4)!} f(n-4),\]
where we ignore cycles of length larger than 4, since we only care about $n = 6$. Substituting $f(0) = f(1) = 1$ and $f(2) = 2$ yields $f(6) = 256$, and so the answer is $\frac{256}{6!} = \boxed{\tfrac{16}{45}}$.

\par Alternatively, we can simply count based on the number of ways to partition 6 as a sum of powers of two, of which there are six, namely $\{(1,1,1,1,1,1),(2,1,1,1,1),(2,2,1,1),(2,2,2),(4,1,1),(4,2)\}$. These give
\[1 + \binom62 + \frac12\binom62\binom42 + \frac16\binom62\binom42 + \binom64\cdot3! + \binom64\cdot3! = 256\]
possible permutations, which is the same as above.





\setcounter{enum}{\theenumi}
\end{enumerate}

\hspace*{-\parindent}%
\begin{minipage}{0.80\textwidth}
\begin{enumerate}
\setcounter{enumi}{\theenum}
\item Ryan stands on the bottom-left square of a 2017 by 2017 grid of squares, where each square is colored either black, gray, or white according to the pattern as depicted to the right. Each second he moves either one square up, one square to the right, or both one up and to the right, selecting between these three options uniformly and independently. Noting that he begins on a black square, find the probability that Ryan is still on a black square after 2017 seconds.
\setcounter{enum}{\theenumi}
\end{enumerate}

\end{minipage} \hfill
\begin{minipage}{0.17\textwidth}
\centering
\vspace{-7pt}
\includegraphics[scale=0.08]{Combo2017-1}
\end{minipage}

\begin{enumerate}
\item[] \proposed{Patrick Lin}

\solution Index the grid by $x$ and $y$ coordinates, and consider the quantity $k = x + y$, such that a square $(x,y)$ is black iff 3 divides $k$. Then in each turn, $k$ increases by 1 with probability 2/3 and by 2 with probability 1/3. We can hence consider the generating function
\[f(k) = \frac{(2k+k^2)^{2017}}{3^{2017}},\]
where we wish to find the sum of the coefficients of the terms with exponents a multiple of 3.

\par Define $\omega = \frac{-1+i\sqrt{3}}{2}$ and observe that computing $\frac13(f(\omega) + f(\omega^2) + f(1))$ will give us the desired answer. Then
\begin{align*}
3^{2017}(f(\omega) + f(\omega^2)) & = (\omega-1)^{2017} + (\omega^2 - 1)^{2017} \\
& = 3^{2017/2}((e^{5\pi i/6})^{2017} + (e^{-5\pi i/6})^{2017}) \\
& = 2 \cdot 3^{2017/2} \cdot \cos(5\pi/6) \\
& = -3^{1009}.
\end{align*}
Since $f(1) = 1$, it follows that the desired answer is
\[\frac13(-3^{-1008} + 1) = \frac13\left(1 - \frac{1}{3^{1008}}\right) = \boxed{\frac{3^{1008}-1}{3^{1009}}}.\]
\end{enumerate}

\end{document}