\documentclass[10pt]{article}
\usepackage{amsmath, amssymb, amsthm}
\usepackage[top=2cm, left = 2cm, right = 2cm, bottom = 3cm]{geometry}
\usepackage[pdftex]{graphicx}
\usepackage{asymptote}
\usepackage{fancyhdr}
\newcommand{\N}{\mathbb{N}}
\newcommand{\lcm}{\operatorname{lcm}}
\pagestyle{fancy}
\rhead{}
\chead{\includegraphics[scale=0.17]{CMIMC-header-2017.png}}
\lhead{}
\setlength{\headheight}{43pt}
\rfoot{}
\cfoot{}
\lfoot{}
\newcommand{\proposed}[1]
{
\vspace{5pt}
\noindent\textit{Proposed by #1}
}
\newcommand{\solution}
{
\vspace{5pt}
\noindent\textit{Solution.}\qquad
}
\begin{document}\thispagestyle{empty}
\begin{center}

\vspace*{90pt}

\includegraphics[scale=0.3]{CMIMC-header-2017.png}

\includegraphics[scale=0.33]{NT-header.png}

\vspace{1.6in}

\includegraphics[scale=0.20]{Instruction-Header.png}
\noindent\rule{17.7cm}{2pt}
\end{center}

\vspace{10pt}

\begin{enumerate}
\large
\item Do not look at the test before the proctor starts the round.

\item This test consists of 10 short-answer problems to be solved in 60 minutes.
	Each question is worth one point.

\item Write your name, team name, and team ID on your answer sheet. Circle the
	subject of the test you are currently taking.

\item Write your answers in the corresponding boxes on the answer sheets.

\item No computational aids other than pencil/pen are permitted.

\item Answers must be reasonably simplified.

\item If you believe that the test contains an error, submit your protest in writing to Doherty 2302 by the end of lunch.
\end{enumerate}
\newpage

\begin{center}
\huge\textbf{Number Theory}\normalsize

\vspace{3pt}
\end{center}

\begin{enumerate}

\item There exist two distinct positive integers, both of which are divisors of $10^{10}$, with sum equal to $157$.  What are they?

\item Determine all possible values of $m+n$, where $m$ and $n$ are positive integers satisfying \[\lcm(m,n) - \gcd(m,n) = 103.\]

\item For how many triples of positive integers $(a,b,c)$ with $1\leq a,b,c\leq 5$ is the quantity \[(a+b)(a+c)(b+c)\] not divisible by $4$?

\item Let $a_1, a_2, a_3, a_4, a_5$ be positive integers such that $a_1, a_2, a_3$ and $a_3, a_4, a_5$ are both geometric sequences and $a_1, a_3, a_5$ is an arithmetic sequence. If $a_3 = 1575$, find all possible values of $\vert a_4 - a_2 \vert$.


\item One can define the greatest common divisor of two positive rational numbers as follows: for $a$, $b$, $c$, and $d$ positive integers with $\gcd(a,b)=\gcd(c,d)=1$, write \[\gcd\left(\dfrac ab,\dfrac cd\right) = \dfrac{\gcd(ad,bc)}{bd}.\] For all positive integers $K$, let $f(K)$ denote the number of ordered pairs of positive rational numbers $(m,n)$ wiht $m<1$ and $n<1$ such that \[\gcd(m,n)=\dfrac{1}{K}.\] What is $f(2017)-f(2016)$?

\item Find the largest positive integer $N$ satisfying the following properties:

\begin{itemize}

\item $N$ is divisible by $7$;

\item Swapping the $i^{\text{th}}$ and $j^{\text{th}}$ digits of $N$ (for any $i$ and $j$ with $i\neq j$) gives an integer which is \textit{not} divisible by $7$.

\end{itemize}

\item The \textit{arithmetic derivative} $D(n)$ of a positive integer $n$ is defined via the following rules:

\begin{itemize}

\item $D(1) = 0$;

\item $D(p)=1$ for all primes $p$;

\item $D(ab)=D(a)b+aD(b)$ for all positive integers $a$ and $b$.

\end{itemize}

Find the sum of all positive integers $n$ below $1000$ satisfying $D(n)=n$.

\item Let $N$ be the number of ordered triples $(a,b,c) \in \{1, \ldots, 2016\}^{3}$ such that $a^{2} + b^{2} + c^{2} \equiv 0 \pmod{2017}$. What are the last three digits of $N$?

\item Find the smallest prime $p$ for which there exist positive integers $a,b$ such that 
	\[
		a^{2} + p^{3} = b^{4}.
	\]

\item For each positive integer $n$, define \[g(n) = \gcd\left\{0! n!, 1! (n-1)!, 2 (n-2)!, \ldots, k!(n-k)!, \ldots, n! 0!\right\}.\] Find the sum of all $n \leq 25$ for which $g(n) = g(n+1)$.
\end{enumerate}
\end{document}
