\documentclass[10pt]{article}
\usepackage{amsmath, amssymb, amsthm}
\usepackage[top=2cm, left = 2cm, right = 2cm, bottom = 3cm]{geometry}
\usepackage[pdftex]{graphicx}
\usepackage{asymptote}
\usepackage{fancyhdr}
\pagestyle{fancy}
\rhead{}
\chead{\includegraphics[scale=0.12]{CMIMC-header.png}}
\lhead{}
\setlength{\headheight}{43pt}
\rfoot{}
\cfoot{}
\lfoot{}
\newcommand{\proposed}[1]
{
\vspace{5pt}
\noindent\textit{Proposed by #1}
}
\newcommand{\solution}
{
\vspace{5pt}
\noindent\textit{Solution.}\qquad
}
\begin{document}

\begin{center}
\huge\textbf{Algebra Solutions Packet}\normalsize

\vspace{3pt}
\end{center}

\begin{enumerate}

\item In a race, people rode either bicycles with blue wheels or tricycles with
tan wheels. Given that 15 more people rode bicycles than tricycles and there
were 15 more tan wheels than blue wheels.  What is the total number of people
who rode in the race?

\proposed{Patrick Lin}

\solution Let $b$ be the number of bicycles ridden and $t$ be the number of tricycles ridden. Then we have
\begin{align*}
b - t & = 15 \\
3t - 2b & = 15.
\end{align*}
Solving yields $b = 60$ and $t = 45$, and thus the answer is $\boxed{105}$.
\item Suppose that some real number $x$ satisfies
\[\log_2 x + \log_8 x + \log_{64} x = \log_x 2 + \log_x 16 + \log_x 128.\]
Given that the value of $\log_2 x + \log_x 2$ can be expressed as $\tfrac{a\sqrt{b}}{c}$, where $a$ and $c$ are coprime and $b$ is squarefree, compute $abc$.

\proposed{Patrick Lin}

\solution Let $\log_2 x = k$. Then simplification yields $\log_2 x + \log_8 x + \log_{64} x = \tfrac32 k$. Similarly, we have $\log_x 2 + \log_x 16 + \log_x 128 = \tfrac{12}{k}$. Setting them equal, we have $k = 2\sqrt{2}$. Then we have $k + \tfrac{1}{k} = \frac{9\sqrt{2}}{4}$, and so the answer is $\boxed{72}$.


\item Let $\ell$ be a real number satisfying the equation
$\tfrac{(1+\ell)^2}{1+\ell^2}=\tfrac{13}{37}$.  Then
\[\frac{(1+\ell)^3}{1+\ell^3}=\frac mn,\] where $m$ and $n$ are positive
coprime integers.  Find $m+n$.

\proposed{David Altizio}

\solution Replace $\tfrac{13}{37}$ with a general $N$.  Note that the given rearranges to \[N=\dfrac{(1+\ell)^2}{1+\ell^2}=1+\dfrac{2\ell}{1+\ell^2}=1+\dfrac{2}{\ell+\frac1{\ell}}\implies \ell+\frac1{\ell}=\frac{2}{N-1}.\] Now remark that \begin{align*}\dfrac{(1+\ell)^3}{1+\ell^3}&=\dfrac{(1+\ell)^3}{(1+\ell)(\ell^2-\ell+1)}=\dfrac{(\ell+1)^2}{\ell^2-\ell+1}\\&=1+\dfrac{3\ell}{\ell^2-\ell+1}=1+\dfrac{3}{\ell+\frac{1}{\ell}-1}.\end{align*} Hence substituting our expression for $\ell+\tfrac1{\ell}$ yields \[1+\dfrac3{\ell+\frac1{\ell}-1}=1+\dfrac{3}{\frac{2}{N-1}-1}=1+\dfrac{3(N-1)}{3-N}=\dfrac{2N}{3-N}.\] Plugging in $N=\tfrac{13}{37}$ and simplifying gives a result of $\tfrac{13}{49}$, so the requested answer is $13+49=\boxed{62}$.

\item A line with negative slope passing through the point $(18,8)$ intersects the $x$ and $y$ axes at $(a,0)$ and $(0,b)$ respectively.  What is the smallest possible value of $a+b$? 

\proposed{David Altizio}

\solution Note that the equation of the line can be written as $y-8=m(x-18)$ for some $m<0$; this is just point-slope form.  For simplicity, let $m_0=-m$, so that $m_0$ is positive; then the equation rewrites to $y-8=m_0(18-x)$.  Substituting $y=0$ yields $a=18+\tfrac{8}{m_0}$, and substituting $x=0$ yields $b=8+18m_0$.  Therefore \[a+b=18+8+\left(\frac{8}{m_0}+18m_0\right)=26+\frac{8}{m_0}+18m_0.\] This can be easily maximized by AM-GM or calculus, but instead we propose an alternate approach.  Let $\tfrac{8}{m_0}+18m_0=K$ for some $K\geq 0$.  Then $18m_0^2-Km_0+8=0$.  We thus seek to find the maximum possible $K$ such that the quadratic $18t^2-Kt+8$ has at least one real solution.  This condition is equivalent to the discriminant being nonnegative, so we must have \[K^2-4\cdot 18\cdot 8 \geq 0\quad\implies\quad K\geq 24.\] Therefore $a+b\geq 26+24=\boxed{50}$.

\item The parabolas $y=x^2+15x+32$ and $x = y^2+49y+593$ are tangent to each other at some point $(x_0,y_0)$.  Find $x_0+y_0$.

\proposed{Andrew Kwon}

\solution Adding the two equations yields $x+y = x^{2} + 15x + y^{2} + 49y +
625$, which is equivalent to $x^{2} + 14x + 49 + y^{2} + 48y + 576 = 0$. This
factors as a sum of squares $(x+7)^{2} + (y+24)^{2} = 0$, and so $x= -7, y =
-24$. Then, $(x_{0},y_{0}) = (-7,-24)$ and $x_{0} + y_{0} = \boxed{-31}$.

\item For some complex number $\omega$ with $|\omega| = 2016$, there is some $\lambda> 1$
	such that $\omega, \omega^{2}, \lambda \omega$ form an equilateral triangle in the
	complex plane. Then, $\lambda$ can be written in the form
	$\frac{a+\sqrt{b}}{c}$, with $a,b,c$ positive integers. Compute
	$\sqrt{a+b+c}$.

\proposed{Andrew Kwon}

\solution In general, let $|\omega| = n$. Note that $\omega, \omega^{2}, \lambda
\omega$ form an equilateral triangle if and only if $1, \lambda, \omega$ for an
equilateral triangle in the complex plane. This is because multiplying each
number by $\omega$ scales and rotates every point in the plane by the same
amount. Interpreting the complex numbers $1, \omega$ as vectors, it follows that
$\lambda$ exists only if the angle between $1, \omega$ is $\frac{\pi}{3}$. Also
note that $| \omega - 1 | = |\lambda - 1|  = \lambda - 1$ since
$\lambda, 1$ are other vertices of the same equilateral triangle. By
the Law of Cosines, we have 
\begin{align*}
    |w|^{2} &= 1 + |\omega - 1|^{2} + |\omega-1| \\
    &= 1 + (\lambda -1)^{2} + (\lambda - 1),
\end{align*}
and so $\lambda = \frac{1 + \sqrt{4n^{2} - 3}}{2}.$ It follows that $a + b + c =
4n^{2},$ and so the desired answer is simply $\sqrt{4032^{2}} = \boxed{4032}.$

\item Let $a$, $b$, $c$, and $d$ be positive real numbers which satisfy the system of equations \begin{align*}(a+b)(c+d)&=143,\\(a+c)(b+d)&=150,\\(a+d)(b+c)&=169.\end{align*} Find the smallest possible value of $a^2+b^2+c^2+d^2$.

\proposed{David Altizio}

\solution Note that the equations expand to \begin{align*}ac+ad+bc+bd&=143,\\ab+ad+bc+cd&=150,\\ab+ac+bd+cd&=169.\end{align*} Adding all these equalities together yields \[2(ab+ac+ad+bc+bd+cd)=143+150+169=462.\] As a result, we have \[(a+b+c+d)^2=a^2+b^2+c^2+d^2+2(ab+ac+ad+bc+bd+cd)=a^2+b^2+c^2+d^2+462.\] Hence in order to minimize $a^2+b^2+c^2+d^2$ it suffices to minimize $a+b+c+d$.

\par To do this, note that by AM-GM on the last equation we have \[(a+d)(b+c)\leq\left(\dfrac{a+b+c+d}2\right)^2\implies (a+b+c+d)^2\geq 4\cdot 169=676.\] This is in fact sufficient to guarantee the existence of $a,b,c,d$ which satisfy all three equations.  To see this, let $s=a+b+c+d$, and note that the original system can be written as \begin{align*}(a+b)(s-(a+b))&=143,\\(a+c)(s-(a+c))&=150,\\(a+d)(s-(a+d))&=169.\end{align*} These are quadratics in $a+b$, $a+c$, and $a+d$ respectively; as a result, whenever $s\geq 26$ the values of $a+b$, $a+c$, and $a+d$ are all real.  Adding these together allows one to solve for $a$, from which the values of the other three variables follow.  (A computer simulation ensures that $a$, $b$, $c$, and $d$ are all positive.)

\par Hence, we have \[a^2+b^2+c^2+d^2=(a+b+c+d)^2-462\geq 676-462=\boxed{214}.\]

\item Let $r_1$, $r_2$, $\ldots$, $r_{20}$ be the roots of the polynomial $x^{20}-7x^3+1$.  If \[\dfrac{1}{r_1^2+1}+\dfrac{1}{r_2^2+1}+\cdots+\dfrac{1}{r_{20}^2+1}\] can be written in the form $\tfrac mn$ where $m$ and $n$ are positive coprime integers, find $m+n$.

\proposed{David Altizio}

\solution Note that by partial fraction decomposition \[\dfrac{1}{r_k^2+1}=\dfrac{1}{(r_k-i)(r_k+i)}=\dfrac{1}{2i}\left(\dfrac{1}{r_k-i}-\dfrac{1}{r_k+i}\right)\] for all positive integers $1\leq k\leq 20$.  This in turn means that the expression we wish to evaluate can be rewritten as \[\frac{1}{2i}\sum_{k=1}^{20}\dfrac1{r_k-i}-\dfrac1{2i}\sum_{k=1}^{20}\dfrac1{r_k+i}.\] We now present two ways to compute this sum: the first one uses standard algebraic techniques, while the second employs a bit of calculus.

\begin{itemize}

\item\textbf{METHOD 1: }Let $Q(x)$ be the polynomial whose roots are $r_1+i$, $r_2+i$, $\ldots$, $r_{20}+i$.  Then by standard methods \[Q(x)=P(x-i)=(x-i)^{20}-7(x-i)^3+1.\] We seek to compute the sum of the reciprocals of the roots of $Q$.  Note that the constant term of $Q$ is $Q(0)=(-i)^{20}-7(-i)^3+1=2-7i$, while by the Binomial Theorem \[[x]Q(x) = \binom{20}1(-i)^{19}-7\binom31(-i)^2=20i+21.\] Thus by Vieta's the sum of the reciprocals of the roots of $Q$ is simply \[\dfrac{-(20i+21)}{2-7i}=\dfrac{98}{53}-\dfrac{182}{53}i.\] To compute the second summation, let $R$ be the polynomial whose roots are $r_1-i$ through $r_{20}-i$.  A similar argument works here as well, but in fact we can save time by noting that for all real $x$ we have \[R(x) = P(x+i) = P(\overline{x-i}) = \overline{P(x-i)}=\overline{Q(x)},\] and so $R(x)\equiv\overline{Q(x)}$ identically.  As a result, all the coefficients of $R$ are conjugates of the coefficients of $Q$, and so the sum of the reciprocals of the roots of $R$ is $\tfrac{98}{53}+\tfrac{182}{53}i$.  Hence the desired sum is \[\dfrac1{2i}\left[\left(\dfrac{98}{53}+\frac{182}{53}i\right)-\left(\frac{98}{53}-\frac{182}{53}i\right)\right]=\dfrac{182}{53}\] and the requested answer is $182+53=\boxed{235}.$

\item\textbf{METHOD 2:} Rewrite the sum as \[\frac{1}{2i}\sum_{k=1}^{20}\dfrac1{(-i)-r_k}-\dfrac1{2i}\sum_{k=1}^{20}\dfrac1{i-r_k}.\] We now make use of the following lemma.

\par\textbf{LEMMA: }Let $P$ be a polynomial of degree $n$ and $r_1,\ldots,r_n$ its roots.  Then for any $x$, \[\sum_{k=1}^n\dfrac{1}{x-r_k}=\dfrac{P'(x)}{P(x)}.\]

\begin{proof}
Assume WLOG that $P$ is monic; we can do this since scaling $P$ by a constant changes neither the roots of $P$ nor the ratio $P'(x)/P(x)$ for any $x$.  Rewrite $P(x)$ as $\prod_{i=1}^{n}(x-r_i)$.  Then remark \[P'(x)=\dfrac{d}{dx}\left[\prod_{i=1}^{n}(x-r_i)\right]=\sum_{i=1}^{n}\prod_{1\leq j\neq i\leq n}(x-r_j).\] This is basically a generalization of the Product Rule for derivatives.  Finally, we can connect this to the sum in question by noting that \[\sum_{i=1}^{n}\dfrac{1}{x-r_i}=\dfrac1{\prod_{i=1}^{n}(x-r_i)}\sum_{i=1}^{n}\prod_{1\leq j\neq i\leq n}(x-x_j)=\dfrac{P'(x)}{P(x)}\] as desired.
\end{proof}
With this, our summation becomes \[\frac{1}{2i}\sum_{k=1}^{20}\dfrac1{(-i)-r_k}-\dfrac1{2i}\sum_{k=1}^{20}\dfrac1{i-r_k}=\dfrac{1}{2i}\left(\dfrac{P'(-i)}{P(-i)}-\dfrac{P'(i)}{P(i)}\right),\] which from our remark in Method 1 is just the imaginary part of $\tfrac{P'(-i)}{P(-i)}$.  Computation yields $\tfrac{182}{53}$, giving the same answer as before.

\end{itemize}

\item Let $\lfloor x\rfloor$ denote the greatest integer function and $\{x\}=x-\lfloor x\rfloor$ denote the fractional part of $x$.  Let $1\leq x_1<\ldots<x_{100}$ be the $100$ smallest values of $x\geq 1$ such that $\sqrt{\lfloor x\rfloor\lfloor  x^3\rfloor}+\sqrt{\{x\}\{x^3\}}=x^2.$ Compute \[\sum_{k=1}^{50}\dfrac{1}{x_{2k}^2-x_{2k-1}^2}.\]

\proposed{Cody Johnson}

\solution By Cauchy-Schwarz, \[\sqrt{\lfloor x\rfloor\lfloor x^3\rfloor}+\sqrt{\{x\}\{x^3\}}\le\sqrt{(\lfloor x\rfloor+\{x\})(\lfloor x^3\rfloor+\{x^3\})}=x^2.\] This in turn means that we actually have equality.  Recall that in the two variable case of Cauchy-Schwarz, \[(a^2+b^2)(c^2+d^2)\geq(ac+bd)^2,\] the equality case holds when $\tfrac ac=\tfrac bd$, or $ad=bc$.  Applying this here yields  \begin{align*}&\qquad\quad\lfloor x\rfloor\{x^3\}=\lfloor x^3\rfloor\{x\}\\&\iff\lfloor x\rfloor x^2=\lfloor x^3\rfloor\\&\iff2\lfloor x\rfloor^2\{x\}+\lfloor x\rfloor\{x\}^2=\left\lfloor3\lfloor x\rfloor^2\{x\}+3\lfloor x\rfloor\{x\}^2+\{x\}^3\right\rfloor.\end{align*} We also have $\lfloor x\rfloor x^2=\lfloor x^3\rfloor>x^3-1$, so $\lfloor x\rfloor>x-\frac1{x^2}$, i.e. $\{x\}<\frac1{x^2}$. This allows us to easily bound \[\lfloor3\lfloor x\rfloor^2\{x\}+3\lfloor x\rfloor\{x\}^2+\{x\}^3\rfloor\in\{0,1,2,3\},\] which means $\lfloor x\rfloor\{x\}(2\lfloor x\rfloor+\{x\})\in\{0,1,2,3\}$. Let this expression equal $r$.  Writing this as a quadratic in $\{x\}$ and using the Quadratic Formula yields \[\{x\}=-\lfloor x\rfloor+\frac{\sqrt{\lfloor x\rfloor^4+r\lfloor x\rfloor}}{\lfloor x\rfloor},\] where $r\in\{0,1,2,3\}$. Since we need this to be in the interval $[0,\frac1{x^2})$, we can further bound to get that these values work precisely when $r\in\{0,1\}$. Thus, the solution set is $x_{2n-1}=n$, $x_{2n}=\sqrt{n^2+\frac1n}$ for $n=1,2,\dots,50$. Finally, we compute \[\sum_{n=1}^{50}\frac1{(\sqrt{n^2+1/n})^2-n^2}=\sum_{n=1}^{50}n=\frac{50(51)}2=\boxed{1275}.\]

\item Denote by $F_0(x)$, $F_1(x)$, $\ldots$ the sequence of Fibonacci polynomials, which satisfy the recurrence $F_0(x)=1$, $F_1(x)=x$, and $F_n(x)=xF_{n-1}(x)+F_{n-2}(x)$ for all $n\geq 2$.\footnote{In reality, the indexes are shifted up by one (so e.g. $F_1(x)=1$), but this interpretation makes the problem statement easier to write since $\deg F_i(x) = i$ for all $i\geq 0$}  It is given that there exist integers $\lambda_0$, $\lambda_1$, $\ldots$, $\lambda_{1000}$ such that \[x^{1000}=\sum_{i=0}^{1000}\lambda_iF_i(x)\] for all real $x$.  For which integer $k$ is $|\lambda_k|$ maximized?

\proposed{David Altizio}

\solution Replace $1000$ with a general $n$.  I claim that for all $n\geq 0$ we have \[x^n=F_n(x)+\sum_{k=1}^{\lfloor n/2\rfloor}(-1)^k\left[\binom nk-\binom n{k-1}\right]F_{n-2k}(x).\] To prove this, we use mathematical induction.  The base case, $n=0$, is easy.  For the inductive step, assume that the identity holds true for some value of $n$.  We now split into cases.

\begin{itemize}

\item\textbf{CASE 1: $n$ is odd.}  Note that multiplying both sides of the equality by $x$ and using the definition of the Fibonacci polynomials yields \begin{align*}x^{n+1}&=xF_n(x)+\sum_{k=1}^{\lfloor n/2\rfloor}(-1)^k\left[\binom nk-\binom n{k-1}\right]xF_{n-2k}(x)\\&=F_{n+1}(x)-F_{n-1}(x)+\sum_{k=1}^{\lfloor n/2\rfloor}(-1)^k\left[\binom nk-\binom n{k-1}\right]\left[F_{n+1-2k}(x)-F_{n-1-2k}(x)\right]\\&=F_{n+1}(x)-F_{n-1}(x)+\sum_{k=1}^{\lfloor n/2\rfloor}(-1)^k\left[\binom nk-\binom n{k-1}\right]F_{n+1-2k}(x)\\&\hspace{15em}-\sum_{k=1}^{\lfloor n/2\rfloor}(-1)^k\left[\binom nk-\binom n{k-1}\right]F_{n-1-2k}(x).\end{align*} Our end goal is to combine these two summations into one.  To accomplish this, note that shifting the indeces of the second summation up by $1$ and moving the $F_{n-1}(x)$ gives \begin{align*}x^{n+1}&=F_{n+1}(x)+\sum_{k=1}^{\lfloor n/2\rfloor}(-1)^k\left[\binom nk-\binom n{k-1}\right]F_{n+1-2k}(x)\\&\hspace{10em}-\left(F_{n-1}(x)+\sum_{k=2}^{\lfloor n/2\rfloor+1}(-1)^{k-1}\left[\binom n{k-1}-\binom n{k-2}\right]F_{n+1-2k}(x)\right)\\&=F_{n+1}(x)+\sum_{k=1}^{\lfloor n/2\rfloor}(-1)^k\left[\binom nk-\binom n{k-1}\right]F_{n+1-2k}(x)\\&\hspace{10em}-\sum_{k=1}^{\lfloor n/2\rfloor+1}(-1)^{k-1}\left[\binom n{k-1}-\binom n{k-2}\right]F_{n+1-2k}(x).\end{align*} Now we are able to combine the summations together.  First, we deal with the case where $n+1-2k\neq 0$.  Note that in this case, it is not hard to see that the coefficient of $F_{n+1-2k}(x)$ is \begin{align*}&\quad(-1)^k\left[\binom nk-\binom n{k-1}\right]-(-1)^{k-1}\left[\binom n{k-1}-\binom n{k-2}\right]\\&=(-1)^k\left[\left(\binom nk-\binom n{k-1}\right)+\left(\binom n{k-1}-\binom n{k-2}\right)\right]\\&=(-1)^k\left[\left(\binom nk+\binom n{k-1}\right)-\left(\binom n{k-1}+\binom n{k-2}\right)\right]\\&=(-1)^k\left[\binom{n+1}k-\binom{n+1}{k-1}\right].\end{align*} The case where $n+1-2k=0$ (i.e. $k=\tfrac{n+1}2$) is a bit trickier.  Note that the only summation that contains an $F_0(x)$ term is the second one.  Thus, we may conclude that the coefficient of $F_0(x)$ in the final expansion is \[(-1)^{(n+1)/2}\left[\binom{n}{(n-1)/2}-\binom{n}{(n-3)/2}\right].\]  In order for our hypothesis to be correct, it suffices to show that this equals \[(-1)^{(n+1)/2}\left[\binom{n+1}{(n+1)/2}-\binom{n+1}{(n-1)/2}\right].\] To prove this, we need to be slightly clever: since $\tfrac{n+1}2+\tfrac{n-1}2=n$, we have $\binom{n}{(n-1)/2}=\binom{n}{(n+1)/2}$.  After this point, we can apply the same logic as we did above to reach the desired conclusion.

\par We have thus shown in this case that \[x^{n+1}=F_{n+1}(x)+\sum_{k=1}^{\lfloor (n+1)/2\rfloor}(-1)^k\left[\binom {n+1}k-\binom {n+1}{k-1}\right]F_{n+1-2k}(x),\] which is what we wanted.

\textbf{CASE 2: $n$ is even.}  This case is actually very similar to the previous case, so we won't show work here; the only difference relates to that edge case we described above.  More specifically, note that we can't say $xF_{n-2k}(x)=F_{n+1-2k}(x)-F_{n-1-2k}(x)$ for $k=n/2$, since then the second term is $F_{-1}(x)$, which is bad.  Instead, we write $xF_0(x)=F_1(x)$. Details are left to the reader.
\end{itemize}

\par We have thus proven the hypothesis true for $n+1$, and so by the Principle of Mathematical Induction we are done. 

\par Now we work on maximizing the $\lambda_k$.  Note that from the above work we may extract the $\lambda_k$ terms to get \[|\lambda_{n-2k}|=\binom nk-\binom n{k-1}\] for all $0\leq k\leq 500$.  There are many different ways to maximize this expression; the following is only one of those ways.  Note that \begin{align*}\dfrac{|\lambda_{n-2(k+1)}|}{|\lambda_{n-2k}|}&=\dfrac{\dbinom{n}{k+1}-\dbinom{n}k}{\dbinom{n}k-\dbinom{n}{k-1}}\\&=\dfrac{\dfrac{n!}{(n-k-1)!(k+1)!}-\dfrac{n!}{(n-k)!k!}}{\dfrac{n!}{(n-k)!k!}-\dfrac{n!}{(n-k+1)!(k-1)!}}\\&=\dfrac{(n-k+1)(n-k)-(n-k+1)(k+1)}{(n-k+1)(k+1)-(k+1)k}\\&=\dfrac{(n-k+1)(n-2k-1)}{(k+1)(n-2k+1)}.\end{align*} Setting this ratio to be less than $1$ and expanding yields \[(n-k+1)(n-2k-1)<(k+1)(n-2k+1)\quad\implies\quad 4k^2-4kn+(n^2-n-2)<0.\] Note that by the Quadratic Formula the solutions to $4k^2-4kn+(n^2-n-2)=0$ are \[k=\dfrac{4n\pm\sqrt{(4n)^2-4(4)(n^2-n-2)}}{2\cdot 4}=\dfrac{n\pm\sqrt{n+2}}2.\] The positive root gives a value of $k$ which is above our range, so the largest value of $k$ for which $|\lambda_{n-2(k+1)}|<|\lambda_{n-2k}|$ is $k=\lfloor\tfrac{n-\sqrt{n+2}}2\rfloor$.  For $n=1000$, this gives $k=484$, and so $|\lambda_{n-2k}|=|\lambda_{32}|$ is the maximum over all $\lambda_i$.  The requested answer is thus $\boxed{32}$.

\end{enumerate}

\end{document}
