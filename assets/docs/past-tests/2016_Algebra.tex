\documentclass[10pt]{article}
\usepackage{amsmath, amssymb, amsthm}
\usepackage[top=2cm, left = 2cm, right = 2cm, bottom = 3cm]{geometry}
\usepackage[pdftex]{graphicx}
\usepackage{asymptote}
\usepackage{fancyhdr}
\pagestyle{fancy}
\rhead{}
\chead{\includegraphics[scale=0.12]{CMIMC-header.png}}
\lhead{}
\setlength{\headheight}{43pt}
\rfoot{}
%\cfoot{Page \thepage}
\cfoot{}
\lfoot{}
\addtolength\footskip{-1cm}
\begin{document}\thispagestyle{empty}
\begin{center}

\vspace*{90pt}

\includegraphics[scale=0.23]{CMIMC-header.png}

\includegraphics[scale=0.35]{algebra-header.png}

\vspace{1.6in}

\includegraphics[scale=0.20]{instruction-header.png}
\noindent\rule{17.7cm}{2pt}
\end{center}

\vspace{10pt}

\begin{enumerate}
\large
\item Do not look at the test before the proctor starts the round.

\item This test consists of 10 short-answer problems to be solved in 60 minutes.
	Each question is worth one point.

\item Write your name, team name, and team ID on your answer sheet. Circle the
	subject of the test you are currently taking.

\item Write your answers in the corresponding boxes on the answer sheets.

\item No computational aids other than pencil/pen are permitted.

\item All answers are integers.

\item If you believe that the test contains an error, submit your protest in writing to Porter 100.
\end{enumerate}
\newpage

\begin{center}
\huge\textbf{Algebra}\normalsize

\vspace{3pt}
\end{center}


\begin{enumerate}
\setlength{\itemsep}{3pt}

\item In a race, people rode either bicycles with blue wheels or tricycles with
tan wheels. Given that 15 more people rode bicycles than tricycles and there
were 15 more tan wheels than blue wheels, what is the total number of people
who rode in the race?

\item Suppose that some real number $x$ satisfies
\[\log_2 x + \log_8 x + \log_{64} x = \log_x 2 + \log_x 16 + \log_x 128.\]
Given that the value of $\log_2 x + \log_x 2$ can be expressed as $\tfrac{a\sqrt{b}}{c}$, where $a$ and $c$ are coprime positive integers and $b$ is squarefree, compute $abc$.

\item Let $\ell$ be a real number satisfying the equation
$\tfrac{(1+\ell)^2}{1+\ell^2}=\tfrac{13}{37}$.  Then
\[\frac{(1+\ell)^3}{1+\ell^3}=\frac mn,\] where $m$ and $n$ are positive
coprime integers.  Find $m+n$.

\item A line with negative slope passing through the point $(18,8)$ intersects the $x$ and $y$ axes at $(a,0)$ and $(0,b)$, respectively.  What is the smallest possible value of $a+b$?

\item The parabolas $y=x^2+15x+32$ and $x = y^2+49y+593$ meet at one point $(x_0,y_0)$.  Find $x_0+y_0$.

\item For some complex number $\omega$ with $|\omega| = 2016$, there is some
	real $\lambda>1$ such that $\omega, \omega^{2},$ and $\lambda \omega$
	form an equilateral triangle in the complex plane. Then, $\lambda$ can be written in the form $\frac{a + \sqrt{b}}{c}$, where $a,b,$ and $c$ are positive integers and $b$ is squarefree. Compute $\sqrt{a+b+c}$.

\item Suppose $a$, $b$, $c$, and $d$ are positive real numbers that satisfy the system of equations \begin{align*}(a+b)(c+d)&=143,\\(a+c)(b+d)&=150,\\(a+d)(b+c)&=169.\end{align*} Compute the smallest possible value of $a^2+b^2+c^2+d^2$.

\item Let $r_1$, $r_2$, $\ldots$, $r_{20}$ be the roots of the polynomial $x^{20}-7x^3+1$.  If \[\dfrac{1}{r_1^2+1}+\dfrac{1}{r_2^2+1}+\cdots+\dfrac{1}{r_{20}^2+1}\] can be written in the form $\tfrac mn$ where $m$ and $n$ are positive coprime integers, find $m+n$.

\item Let $\lfloor x\rfloor$ denote the greatest integer function and $\{x\}=x-\lfloor x\rfloor$ denote the fractional part of $x$.  Let $1\leq x_1<\ldots<x_{100}$ be the $100$ smallest values of $x\geq 1$ such that $\sqrt{\lfloor x\rfloor\lfloor  x^3\rfloor}+\sqrt{\{x\}\{x^3\}}=x^2.$ Compute \[\sum_{k=1}^{50}\dfrac{1}{x_{2k}^2-x_{2k-1}^2}.\]

\item Denote by $F_0(x)$, $F_1(x)$, $\ldots$ the sequence of Fibonacci polynomials, which satisfy the recurrence $F_0(x)=1$, $F_1(x)=x$, and $F_n(x)=xF_{n-1}(x)+F_{n-2}(x)$ for all $n\geq 2$. %\footnote{In reality, the indices are shifted up by one (so e.g. $F_1(x)=1$), but this interpretation makes the problem statement easier to write since $\deg F_i(x) = i$ for all $i\geq 0$}  
It is given that there exist unique integers $\lambda_0$, $\lambda_1$, $\ldots$, $\lambda_{1000}$ such that \[x^{1000}=\sum_{i=0}^{1000}\lambda_iF_i(x)\] for all real $x$.  For which integer $k$ is $|\lambda_k|$ maximized?

\end{enumerate}
\end{document}
