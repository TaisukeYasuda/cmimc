\documentclass[10pt]{article}
\usepackage{amsmath, amssymb, amsthm}
\usepackage[top=2cm, left = 2cm, right = 2cm, bottom = 3cm]{geometry}
\usepackage[pdftex]{graphicx}
\usepackage{asymptote}
\usepackage{fancyhdr}
\pagestyle{fancy}
\rhead{}
\chead{\includegraphics[scale=0.12]{CMIMC-header.png}}
\lhead{}
\setlength{\headheight}{43pt}
\rfoot{}
\cfoot{}
\lfoot{}
\newcommand{\proposed}[1]
{
\vspace{5pt}
\noindent\textit{Proposed by #1}
}
\newcommand{\solution}
{
\vspace{5pt}
\noindent\textit{Solution.}\qquad
}
\begin{document}

\begin{center}
\huge\textbf{Number Theory Solutions}\normalsize

\vspace{3pt}
\end{center}

\begin{enumerate}
\setlength{\itemsep}{3pt}

\item David, when submitting a problem for CMIMC, wrote his answer as $100\tfrac
	xy$, where $x$ and $y$ are two positive integers with $x<y$. Andrew
	interpreted the expression as a product of two rational numbers, while
	Patrick interpreted the answer as a mixed fraction.  In this case,
	Patrick's number was exactly double Andrew's!  What is the smallest
	possible value of $x+y$? \\

	\proposed{David Altizio}

	\solution According to the problem statement, Andrew interpreted David's
	result as $\frac{100x}{y}$, while Patrick interpreted it as $100 +
	\frac{x}{y}$. Since Patrick's number was twice as large as Andrew's we
	have
	\[
		\frac{200x}{y} = 100 + \frac{x}{y} \implies
		\frac{x}{y} = \frac{100}{199}.
	\]
	Therefore the smallest possible value of $x+y$ is $\boxed{299}$, achieved when
	$x=100$ and $y=199$.



\item Let $a_1$, $a_2$, $\ldots$ be an infinite sequence of integers such that
	$k$ divides $\gcd(a_{k-1},a_k)$ for all $k\geq 2$.  Compute the smallest
	possible value of $a_1+a_2+\cdots+a_{10}$. \\%David Altizio

	\proposed{David Altizio}

	\solution Note that the condition implies that $a_{k}$ is divisible by
	both $k,k+1$ for all $k \geq 1$. In particular, $a_{k} \geq k(k+1)$.
	Also, the construction $a_{k} = k(k+1)$ will satisfy the conditions of
	the problem, so the smallest possible value of the sum $a_{1} + \ldots +
	a_{10}$ is
	\[
		1 \cdot 2 + 2 \cdot 3 + \ldots + 10\cdot 11 = \boxed{440}.
	\]

\item How many pairs of integers $(a,b)$ are there such that $0\leq a < b \leq 100$ and such that $\tfrac{2^b-2^a}{2016}$ is an integer? %Cody Johnson

	\proposed{Cody Johnson}

	\solution Factoring 2016 as $2^{5} \cdot 3^{2} \cdot 7$, it follows that
	$2^{5} | 2^{b} - 2^{a}$, whence $a \geq 5$, and also $9 | 2^{b} -
	2^{a}$, whence $6 | b-a$. Consider $b - a = 6n$ for some positive
	integer $n$. Then, $5 \leq a \leq 100 - 6n$, and so there are $96 - 6n$
	possible values of $a$ with precisely one corresponding value of $b$ for
	a given $n$. Note that $n > 0$ because $b > a$. Thus, the number of pairs 
	can be counted by
	\[
		\sum_{n=1}^{16} 96 - 6n = 96 \cdot 16 - 16 \left(\frac{16 \cdot
		17}{2}\right)
	\]
	which evaluates to $\boxed{720}$.

\item For some positive integer $n$, consider the usual prime factorization \[n =
	\displaystyle \prod_{i=1}^{k} p_{i}^{e_{i}}=p_1^{e_1}p_2^{e_2}\ldots p_k^{e_k},\] where $k$ is the number of
	primes factors of $n$ and $p_{i}$ are the prime factors of $n$. Define
	$Q(n), R(n)$ by
	\[
		Q(n) = \prod_{i=1}^{k} p_{i}^{p_{i}} \text{ and } R(n) =
		\prod_{i=1}^{k} e_{i}^{e_{i}}.
	\]
	For how many $1 \leq n \leq 70$ does $R(n)$ divide $Q(n)$?\\

\proposed{Andrew Kwon}

\solution I claim that, by counting the complement, only the $n$ with 
$e_{i} \neq 1, p_{i}$ need be considered. Indeed, if $e_{i} = 1, p_{i}$ for 
all $i$ then it is evident that $R(n) | Q(n)$. Now, we consider the multiples 
of 8,9,25, or 49 less than 70, since this is a superset of the possible $n$ 
with some $e_{i} \neq 1,p_{i}$. Note that these are all disjoint. 
\begin{itemize}
	\item For multiples of 8, the multiples 8, 16, 32, 40, 48, 56, 64 fail,
		which contributes 7 failures.
	\item For multiples of 9, the multiples 9, 18, 36, 45, 54, 63 fail,
		which contributes 6 failures.
	\item For multiples of 25, the multiple 25, 50 fail, which contributes 2
		failures.
	\item For multiples of 49, the multiple 49 fails, which contributes 1
		failure.
\end{itemize}
These are the only integers which fail from 2 to 70, of which there are 16.
The number $1$ also works (an empty product by default evaluates to $1$.)  Thus, there are 69-16+1 = \boxed{54} integers $n$ such that $R(n)$ divides $Q(n)$.

\item Determine the sum of the positive integers $n$ such that there exist primes
	$p,q,r$ satisfying $p^{n} + q^{2} = r^{2}$. \\

\proposed{Andrew Kwon}

\solution By parity, one of the primes must be $2$, while $r \neq 2$. \\
\par First consider the case when $p = 2$. Then, $2^{n} = r^{2} - q^{2} = (r-q)(r+q)$, 
and so $r-q, r+q$ are powers of 2, say $2^{a}, 2^{b},$ with $0 \leq a < b$. 
Then, $ r = \displaystyle \frac{1}{2}( 2^{a} + 2^{b} )$. If $a = 0$, then $r$ is
not an integer; if $a > 1$, then $b > a > 1$, and $r$ is even. Neither of these
are possible, and so $a = 1$. Thus we can write $r = 2^{b-1} + 1, q =
2^{b-1} -1 $. Since $2^{b-1} \equiv \pm 1 \pmod{3}$, it follows that one of
$r,q$ must be divisible by 3; $r =3 \implies q =1$, is impossible, and so we find
that $q = 3, r=5$ is a possible solution with $b = 3$. In this case we find that
$n = 4$.\\

\par Otherwise, suppose $q = 2$. Then, $p^{n} = (r-2)(r+2)$, and as before we
may write $r-2 = p^{a}, r+2 = p^{b}$. Then, $r = \displaystyle
\frac{1}{2} ( p^{a} + p^{b} )$ but $p \not\mid r \implies a = 0$. Now, $2
=\displaystyle \frac{1}{2} (p^{b} -1)$, and so $ p = 5, b = 1, r = 3$, and $n =
1$. \\

These are the only solutions, and so the sum of the possible $n$ is \boxed{5}.

\item Define a \textit{tasty residue} of $n$ to be an integer $1\leq a\leq n$ such that there exists an integer $m>1$ satisfying \[a^m\equiv a\pmod n.\] Find the number of tasty residues of $2016$. \\

\proposed{Andrew Kwon}

\solution The number of tasty residues of $n = p_i^{e_i} \cdots p_k^{e_k}$ is
\[ \prod_{i=1}^k ( \varphi(p_i^{e_i}) + 1) .\]
Indeed, we need $p_i^{e_i} | a^m - a$ for some $m > 1$. For each of these relatively prime moduli, this can occur only in $\varphi(p_i^{e_i}) +1$ ways; either $a$ is relatively prime to $p_i$, or $p_i^{e_i} | a$. Thus, by the Chinese Remainder Theorem there are 
\[ \prod_{i=1}^{k} (\varphi(p_i^{e_i})+1) \]
total solutions modulo $n$. For $n=2016$, this evaluates to $\boxed{833}$.


\item Determine the smallest positive prime $p$ which satisfies the congruence
	\[p+p^{-1}\equiv 25\pmod{143}.\] Here, $p^{-1}$ as usual denotes
	multiplicative inverse. \\%David Altizio

	\proposed{David Altizio}

	\solution Multiply both sides of the equivalence by $p$ to obtain
	$p^{2} + 1 \equiv 25p \pmod{143}$. This means that
	\[
		p^{2} - 25p + 1 \equiv p^{2} - 25p + 144 \equiv (p-9)(p-16)
		\equiv 0 \pmod{143}.
	\]
	Note that $p=9, 16$ are trivially solutions to this congruence, but
	there are other ones as well. In particular, note that $p-9 \equiv 0
	\pmod{11}$ and $p - 16 \equiv 0 \pmod{13}$ gives $p \equiv 42
	\pmod{143}$, while $p-9 \equiv 0 \pmod{13}$ and $p-16 \equiv 0
	\pmod{11}$ gives $p \equiv 126 \pmod{143}$. 
	\par \noindent Now the rest of the problem is straightforward. Remark
	that $9, 16, 42, 126$ are all composite, so we add $143$ to each of
	these residues to get the next set of possible primes: 152, 159, 185,
	269. The first three can be shown to be composite, while \boxed{269} is
	prime, and the smallest possible prime satisfying these conditions.

\item Given that
	\[
		\sum_{x=1}^{70}  \sum_{y=1}^{70} \frac{x^{y}}{y} =
		\frac{m}{67!}
	\]
	for some positive integer $m$, find $m \pmod{71}$.\\

\proposed{Andrew Kwon}

\solution Consider $\displaystyle \sum_{x=1}^{70} \frac{x^{y}}{y}$ for a fixed
$y, 1 \leq y \leq 69.$ Because $71$ is prime, it has some primitive root,
say $r$, and $\{1,r, \ldots, r^{69}\}$ is the set of all residues modulo 71.
It follows that 
	\[
	\sum_{x=1}^{70} x^{y} \equiv \sum_{n=0}^{70} r^{ny} \pmod{71}.
	\]
	However, the right hand side is a geometric series in $r$, which we
	evaluate to be $\displaystyle \frac{r^{71y}-1}{r^{y}-1},$ where we formally
	treat division as multiplication by multiplicative inverses modulo
	71; note that $y<70\implies r^{y}-1 \not\equiv 0 \pmod{71}$, and
	so the above expression is well-defined modulo 71. Thus,
	\[
		\sum_{x=1}^{70} x^{y} \equiv 
		\frac{r^{71y}-1}{r^{y}-1} \pmod{71},
	\]
	while $r^{71y}-1 \equiv 0 \pmod{71}$. Thus, for each $1 \leq y \leq
	69$, the numerator of 
	\[
		\sum_{x=1}^{70} \frac{x^{y}}{y}
	\]
	is divisible by 71. On the other hand, the case where $y = 70$
	yields
	\[
		\sum_{x=1}^{70} x^{70} \equiv 70\pmod{71}.
	\]
	Now, for each $1 \leq y \leq 69$, we have
	\[
		67! \sum_{x=1}^{70} \frac{x^{y}}{y}
	\]
	is an integer, and is divisible by $71$. Thus, these terms do not
	contribute to $m \pmod{71}$. Finally, we consider 
	\[
		67! \sum_{x=1}^{70} \frac{x^{70}}{70},
	\]
	which is also an integer, and so
	\begin{align*}
		m &\equiv 67! \cdot 70^{-1} \sum_{x=1}^{70} x^{70} \pmod{71}\\
		&\equiv 67! \pmod{71}.
	\end{align*}
	Given Wilson's Theorem, it's evident that 
	\begin{align*}
		m \cdot  68 \cdot 69 \cdot 70 &\equiv 70! \pmod{71}\\
		\implies 6m &\equiv 1 \pmod{71},\\
	\end{align*}
	and so $m \equiv \boxed{12} \pmod{71}$.


\item Compute the number of positive integers $n \leq 50$ such that there exist
	distinct positive integers $a,b$ satisfying 
	\[
		\frac{a}{b} +\frac{b}{a} = n \left(\frac{1}{a} + \frac{1}{b}\right).
	\]
	%David Altizio
	\proposed{David Altizio, solution by Andrew Kwon}

	\solution Multiplying both sides of the equation by $ab$ yields 
	\[
		a^{2} + b^{2} = n(a+b).
	\]
	Now, $a^{2} + b^{2} \equiv 0 \pmod{a+b}$, and so $ab \equiv 0
	\pmod{a+b}$, and also $a^{2} \equiv 0 \pmod{a+b}$. Now, let $d =
	\gcd(a,b)$ so that $a = d a', b = db'$, with $a', b'$ relatively prime.
	Then, $a+b | a^{2}$ is equivalent to $a'+b' | d(a')^{2}$. However,
	$a'+b'$ cannot divide $(a')^{2}$, and thus $a'+b' | d$. Finally,
	\[
		n = \frac{a^{2} + b^{2}}{a+b} = \frac{d}{a'+b'} ((a')^{2} +
		(b')^{2}).
	\]
	In particular, $\dfrac{d}{a'+b'}$ must be an integer, and so it follows
	that $n$ is any multiple of a sum of relatively prime squares. It is
	well-known that any prime dividing a sum of squares must be $1
	\pmod{4}$, and so $n$ need only have a prime factor that is $1
	\pmod{4}$. The primes that satisfy this less than 50 are $5, 13, 17, 29,
	37, 41$, and they contribute $10, 3, 2, 1,1,1$ possible $n$
	respectively. Thus, the total possible number of $n$ is $\boxed{18}$.


\item Let $f:\mathbb{N}\mapsto\mathbb{R}$ be the function
	\[f(n)=\sum_{k=1}^\infty\dfrac{1}{\operatorname{lcm}(k,n)^2}.\] It is
	well-known that $f(1)=\tfrac{\pi^2}6$.  What is the smallest positive
	integer $m$ such that $m\cdot f(10)$ is the square of a rational
	multiple of $\pi$? \\%Cody Johnson

	\proposed{Cody Johnson}

	\solution For $d\in\{1,2,5,10\}$, let $S_d:=\displaystyle\sum_{\gcd(k,10)=d}\frac1{k^2}$ and $T_d:=\displaystyle\sum_{d\mid k}\frac1{k^2}=\sum_{k=1}^\infty\frac1{(kd)^2}=\frac{\pi^2}{6d^2}$. Then we have
	\[\sum_{k=1}^\infty\frac1{\text{lcm}(k,10)^2}=\sum_{k=1}^\infty\frac{\gcd(k,10)^2}{k^2\cdot10^2}=\frac1{10^2}\left[1^2\cdot S_1+2^2\cdot S_2+5^2\cdot S_5+10^2\cdot S_{10}\right]\]
	Note that $T_{10}=S_{10}$, $T_5=S_5+S_{10}=S_5+T_{10}$, $T_2=S_2+S_{10}=S_2+T_{10}$, and $T_1=S_1+S_2+S_5+S_{10}=S_1+T_2+T_5-T_{10}$. Therefore, the sum evaluates to
	\[\frac1{10^2}\left[1^2\cdot(T_1-T_2-T_5+T_{10})+2^2\cdot(T_2-T_{10})+5^2\cdot(T_5-T_{10})+10^2\cdot T_{10}\right]=\frac{343\pi^2}{60000}=\frac{7^3\pi^2}{6\cdot10^4}\]
	Thus, $m=6\cdot7=\boxed{42}$.
	
\end{enumerate}

\end{document}

