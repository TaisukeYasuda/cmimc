\documentclass[10pt]{article}
\usepackage{amsmath, amssymb, amsthm}
\usepackage[top=2cm, left = 2cm, right = 2cm, bottom = 3cm]{geometry}
\usepackage[pdftex]{graphicx}
\usepackage{asymptote}
\usepackage{fancyhdr}
\newcommand{\N}{\mathbb{N}}
\pagestyle{fancy}
\rhead{}
\chead{\includegraphics[scale=0.17]{CMIMC-header-2017.png}}
\lhead{}
\setlength{\headheight}{43pt}
\rfoot{}
\cfoot{}
\lfoot{}
\newcommand{\proposed}[1]
{
\vspace{5pt}
\noindent\textit{Proposed by #1}
}
\newcommand{\solution}
{
\vspace{5pt}
\noindent\textit{Solution.}\qquad
}
\begin{document}\thispagestyle{empty}
\begin{center}

\vspace*{90pt}

\includegraphics[scale=0.3]{CMIMC-header-2017.png}

\includegraphics[scale=0.35]{geometry-header.png}

\vspace{1.6in}

\includegraphics[scale=0.20]{Instruction-Header.png}
\noindent\rule{17.7cm}{2pt}
\end{center}

\vspace{10pt}

\begin{enumerate}
\large
\item Do not look at the test before the proctor starts the round.

\item This test consists of 10 short-answer problems to be solved in 60 minutes.
	Each question is worth one point.

\item Write your name, team name, and team ID on your answer sheet. Circle the
	subject of the test you are currently taking.

\item Write your answers in the corresponding boxes on the answer sheets.

\item No computational aids other than pencil/pen are permitted.

\item Answers must be reasonably simplified.

\item If you believe that the test contains an error, submit your protest in writing to Doherty 2302 by the end of lunch.
\end{enumerate}
\newpage

\begin{center}
\huge\textbf{Geometry}\normalsize

\vspace{3pt}
\end{center}

\begin{enumerate}

\item Let $ABC$ be a triangle with $\angle BAC=117^\circ$.  The angle bisector of $\angle ABC$ intersects side $AC$ at $D$.  Suppose $\triangle ABD\sim\triangle ACB$.  Compute the measure of $\angle ABC$, in degrees.

\item Triangle $ABC$ has an obtuse angle at $\angle A$.  Points $D$ and $E$ are placed on $\overline{BC}$ in the order $B$, $D$, $E$, $C$ such that $\angle BAD=\angle BCA$ and $\angle CAE=\angle CBA$.  If $AB=10$, $AC=11$, and $DE=4$, determine $BC$.

\item In acute triangle $ABC$, points $D$ and $E$ are the feet of the angle bisector and altitude from $A$ respectively.  Suppose that $AC - AB = 36$ and $DC - DB = 24$.  Compute $EC - EB$.

\item Let $\mathcal S$ be the sphere with center $(0,0,1)$ and radius $1$ in $\mathbb R^3$.  A plane $\mathcal P$ is tangent to $\mathcal S$ at the point $(x_0,y_0,z_0)$, where $x_0$, $y_0$, and $z_0$ are all positive.  Suppose the intersection of plane $\mathcal P$ with the $xy$-plane is the line with equation $2x+y=10$ in $xy$-space.  What is $z_0$?

\item Two circles $\omega_1$ and $\omega_2$ are said to be $\textit{orthogonal}$ if they intersect each other at right angles.  In other words, for any point $P$ lying on both $\omega_1$ and $\omega_2$, if $\ell_1$ is the line tangent to $\omega_1$ at $P$ and $\ell_2$ is the line tangent to $\omega_2$ at $P$, then $\ell_1\perp \ell_2$.  (Two circles which do not intersect are not orthogonal.)

\par Let $\triangle ABC$ be a triangle with area $20$.  Orthogonal circles $\omega_B$ and $\omega_C$ are drawn with $\omega_B$ centered at $B$ and $\omega_C$ centered at $C$.  Points $T_B$ and $T_C$ are placed on $\omega_B$ and $\omega_C$ respectively such that $AT_B$ is tangent to $\omega_B$ and $AT_C$ is tangent to $\omega_C$.  If $AT_B = 7$ and $AT_C = 11$, what is $\tan\angle BAC$?

\item Cyclic quadrilateral $ABCD$ satisfies $\angle ABD = 70^\circ$, $\angle ADB=50^\circ$, and $BC=CD$.  Suppose $AB$ intersects $CD$ at point $P$, while $AD$ intersects $BC$ at point $Q$.  Compute $\angle APQ-\angle AQP$.

\item Two non-intersecting circles, $\omega$ and $\Omega$, have centers $C_\omega$ and $C_\Omega$ respectively.  It is given that the radius of $\Omega$ is strictly larger than the radius of $\omega$.  The two common external tangents of $\Omega$ and $\omega$ intersect at a point $P$, and an internal tangent of the two circles intersects the common external tangents at $X$ and $Y$.  Suppose that the radius of $\omega$ is $4$, the circumradius of $\triangle PXY$ is $9$, and $XY$ bisects $\overline{PC_\Omega}$.  Compute $XY$.

\item In triangle $ABC$ with $AB=23$, $AC=27$, and $BC=20$, let $D$ be the foot of the $A$ altitude.  Let $\mathcal{P}$ be the parabola with focus $A$ passing through $B$ and $C$, and denote by $T$ the intersection point of $AD$ with the directrix of $\mathcal P$.  Determine the value of $DT^2-DA^2$. (Recall that a parabola $\mathcal P$ is the set of points which are equidistant from a point, called the \textit{focus} of $\mathcal P$, and a line, called the \textit{directrix} of $\mathcal P$.)

\item Let $\triangle ABC$ be an acute triangle with circumcenter $O$, and let $Q\neq A$ denote the point on $\odot (ABC)$ for which $AQ\perp BC$. The circumcircle of $\triangle BOC$ intersects lines $AC$ and $AB$ for the second time at $D$ and $E$ respectively. Suppose that $AQ$, $BC$, and $DE$ are concurrent. If $OD=3$ and $OE=7$, compute $AQ$.

\item Suppose $\triangle ABC$ is such that $AB=13$, $AC=15$, and $BC=14$.  It is given that there exists a unique point $D$ on side $\overline{BC}$ such that the Euler lines of $\triangle ABD$ and $\triangle ACD$ are parallel.  Determine the value of $\tfrac{BD}{CD}$.  (The \textit{Euler} line of a triangle $ABC$ is the line connecting the centroid, circumcenter, and orthocenter of $ABC$.)

%\item Triangle $ABC$ has incircle $\omega$, circumcircle $\Omega$, and circumcenter $O$.  A line $\ell$ tangent to $\omega$ at $T$ perpendicular to $AO$ intersects $\Omega$ at two distinct points $X$ and $Y$.  Supppose that the radius of $\omega$ is $4$, the radius of $\Omega$ is $9$, and $TX\cdot TY = 48$.  Compute the area of $\triangle ABC$.

\end{enumerate}

\end{document}

%%%%%%%%%%%%%%%%%%%%%%%%%%%%%%%%%%%%%%%%%%%%%%%%%%%%%%%%%%%%%%%%%%%%%%%%%%

\item Among the vertices of a cube in $\mathbb R^3$, one has $z$-coordinate zero, one has $z$-coordinate $2016$, three have $z$-coordinate $z_1$, and three have $z$-coordinate $z_2$, where $0<z_1<z_2<2016$.  Find $z_1$.

\proposed{Phillip Wang}

\solution Scale down by a factor of $2016$.  Note that if $B$, $C$, and $D$ are the vertices of the cube adjacent to a vertex $A$, then $z_1$ is the distance from $A$ to the plane formed by $BCD$.  Remark that since the space diagonal of the cube has $1$, the side length of the cube is $\tfrac1{\sqrt 3}$.

\par Now let $E$ be the vertex diametrically opposite $A$.  We claim that $AE$ is perpendicular to the plane $BCD$.  To see this, we need a lemma.\\
\\
\textbf{LEMMA: }Let $SABC$ be a tetrahedron with $SA=SB=SC$.  Then the perpendicular from $S$ to the plane containing $A$, $B$, and $C$ intersects said plane at the circumcenter of $\triangle ABC$.

\begin{proof}Let $O_1$ be the foot of the projection from $S$ to $\triangle ABC$. Then if $SA=SB=SC=s$, one can easily show that \[O_1A=O_1B=O_1C=\sqrt{s^2-O_1S^2},\] so $O_1$ is the circumcenter of $\triangle ABC$ as desired.\end{proof}

Now it is easy to see that $AB=AC=AD$ and $EB=EC=ED$, so the perpendiculars from $A$ and $E$ to $BCD$ coincide, i.e. $AE$ is perpendicular to plane $BCD$.

\par Finally, note that $\triangle BCD$ is an equilateral triangle of side length $\sqrt{\frac 23}$, and thus has area $\tfrac{\sqrt 3}6$.  Thus, the distance $d$ from $A$ to $BCD$ satisfies \[\dfrac13\cdot\dfrac{\sqrt 3}6d = [ABCD] = \dfrac{\sqrt 3}{54}\quad\implies\quad d = \frac13.\] Scaling back up yields a final answer of $\boxed{672}$.

Triangle $ABC$ has incircle $\omega$, circumcircle $\Omega$, and circumcenter $O$.  A line $\ell$ tangent to $\omega$ at $T$ perpendicular to $AO$ intersects $\Omega$ at two distinct points $X$ and $Y$.  Supppose that the radius of $\omega$ is $4$, the radius of $\Omega$ is $9$, and $TX\cdot TY = 48$.  Compute the area of $\triangle ABC$.

\proposed{David Altizio}

\solution Throughout this solution, we use standard triangle notation, so that $K$ is the area of $\triangle ABC$, $R$ is its circumradius, and so on.

\par Let $M$ denote the midpoint of $\overline{XY}$.  Note that by Power of a Point, \begin{align*}TX\cdot TY&=R^2 - OT^2\\&=R^2 - (MT^2+TO^2)\\&=R^2-\left(OI^2-(OM-r)^2\right) - OM^2\\&=R^2-OI^2+(OM-r)^2-OM^2\\&=R^2-(R^2-2Rr)+(OM^2-2OM\cdot r+r^2)-OM^2\\&=2Rr - 2OM\cdot r + r^2.\end{align*}  Setting this equal to $48$ and solving yields $OM = 5$, and since $AO\perp \ell$, we have $AM = R-OM = 4$.

\par Now set $P = AB\cap\ell$ and $Q=AC\cap\ell$.  Remark that the fact that $AO\perp\ell$ gives that $\triangle APQ\sim\triangle ACB$.  Since $\omega$ is the $A$-excircle of $\triangle APQ$, we thus get that the ratio of similitude of the two triangles is $r_a/r$.  As a result, since the distance from $A$ to $PQ$ is $4$, the distance from $A$ to $BC$ must be $4\tfrac{r_a}r = r_a$.

