\documentclass[10pt]{article}
\usepackage{amsmath, amssymb, amsthm}
\usepackage[top=2cm, left = 2cm, right = 2cm, bottom = 3cm]{geometry}
\usepackage[pdftex]{graphicx}
\usepackage{asymptote}
\usepackage{fancyhdr}
\newcommand{\N}{\mathbb{N}}
\newcommand{\lcm}{\operatorname{lcm}}
\pagestyle{fancy}
\rhead{}
\chead{\includegraphics[scale=0.17]{CMIMC-header-2017.png}}
\lhead{}
\setlength{\headheight}{43pt}
\rfoot{}
\cfoot{}
\lfoot{}
\newcommand{\proposed}[1]
{
\vspace{5pt}
\noindent\textit{Proposed by #1}
}
\newcommand{\solution}
{
\vspace{5pt}
\noindent\textit{Solution.}\qquad
}
\begin{document}

\begin{center}
\huge\textbf{Number Theory Solutions Packet}\normalsize

\vspace{3pt}
\end{center}

\begin{enumerate}

	\item There exist two distinct positive integers, both of which are divisors of $10^{10}$, with sum equal to $157$.  What are they?
	
	\proposed{David Altizio}
	
	\solution Suppose $157=x+y$ for $x$ and $y$ divisors of $10^{10}$.  Note that one of $x$ or $y$ must be odd and hence a power of $5$.  Similarly, one of $x$ or $y$ must be not divisible by $5$, and hence a power of $2$.  Thus $157 = 2^a + 5^b$ for some nonnegative integers $a$ and $b$.  Now the largest power of $5$ smaller than $157$ is $125$, and testing a few cases we indeed find that $157-125 = 32$ is the only solution which works.  Thus the two integers are $\boxed{125\text{ and }32}$.

	\item Determine all possible values of $m+n$, where $m$ and $n$ are positive integers satisfying \[\lcm(m,n) - \gcd(m,n) = 103.\]
	
	\proposed{David Altizio}
	
	\solution Recall that by definition the least common multiple of two numbers is a multiple of their $\gcd$.  Let $\lcm(m,n) = k\cdot\gcd(m,n)$ for some positive integer $k$.  Then \[\lcm(m,n) - \gcd(m,n) = k\cdot\gcd(m,n) - \gcd(m,n) = \gcd(m,n)(k-1) = 103.\] Recall that $103$ is prime, so either $\gcd(m,n) = 103$ and $k=2$ or $\gcd(m,n) = 1$ and $k=104$.  In the former case, let $m=103m_0$ and $n=103n_0$.  Then \[\lcm(103m_0,103n_0) = 103\lcm(m_0,n_0) = 103\cdot 2,\] so $\lcm(m_0,n_0) = 2$.  Combining this with the fact that $m\neq n$ means that $m_0$ and $n_0$ must be $1$ and $2$ in some order, i.e. $\{m,n\}=\{103,206\}$.  In the latter case, write $104 = 2^3\cdot 13$.  Since $\gcd(m,n) = 1$, it follows that $m$ and $n$ must either be $1$ and $104$ or $8$ and $13$ in some order.  Combining both of these cases yields that $m+n$ must be either $\boxed{21,\,105,\text{ or }309}$.
	
	\item For how many triples of positive integers $(a,b,c)$ with $1\leq a,b,c\leq 5$ is the quantity \[(a+b)(a+c)(b+c)\] not divisible by $4$?
	
	\proposed{David Altizio}
	
	\solution Note that since the sum of the three multiplicands is $(a+b)+(b+c)+(c+a) = 2(a+b+c)$, we know that at least one of $a+b$, $b+c$, or $c+a$ is even.  Thus the product is always divisible by $2$.  In order for the product to not be divisible by $4$, it must be the case that two of these quantities are odd and the third one is congruent to $2$ modulo $4$.

\par WLOG suppose that $a+b$ and $a+c$ are odd and $b+c\equiv 2\pmod 4$.  Since $(a+b)-(a+c) = b-c$ is even, it follows that $b$ and $c$ are of the same parity.  We now split into cases based on whether both are even or both are odd.

\begin{itemize}

\item If both are even, then they cannot both be congruent modulo $4$, or else their sum would be divisible by $4$.  It follows that $b$ and $c$ must be $4$ and $2$ in some order.  Then $a$ can be $1$, $3$, or $5$; this gives a total of $2\times 3 = 6$ possibilities in this case.

\item If both are odd, then they both must be congruent modulo $4$, or else their sum would be $1+3\equiv 0\pmod 4$.  This means they must be both either $1\pmod 4$ or $3\pmod 4$.  Then $a$ can be either $2$ or $4$, so there are a total of $2\times (2^2+1) = 10$ possibilities in this case.

\end{itemize}

Multiplying by $3$ from our WLOG above gives the final answer as $3(6+10) = \boxed{48}$.

\item Let $a_1, a_2, a_3, a_4, a_5$ be positive integers such that $a_1, a_2, a_3$ and $a_3, a_4, a_5$ are both geometric sequences and $a_1, a_3, a_5$ is an arithmetic sequence. If $a_3 = 1575$, find all possible values of $\vert a_4 - a_2 \vert$.

\proposed{Patrick Lin}

\solution Write the terms as
\[(a_1, a_2, a_3, a_4, a_5) = \left(\frac{m^2}{n^2}a, \frac{m}{n}a, a, \frac{p}{q}a, \frac{p^2}{q^2}a\right),\]
where $m/n$ and $p/q$ are reduced fractions and $a = 1575$. Then arithmetic sequence gives
\[m^2 q^2 + n^2 p^2 = 2n^2q^2.\]
Since $m$ and $n$ are coprime, it follows that $q \mid n$. Similarly, $n \mid q$ and hence $n = q$. We can rewrite
\[m^2 + p^2 = 2n^2.\]
Because each term is an integer, we also have $n^2 \mid a$, and hence $n = 1, 3, 5, 15$, since $1575 = 3^2\cdot 5^2\cdot 7$. Assume that $m\le p$; then the only triples $(m,p,n)$ that satisfy these conditions are 
\[(m,p,n) = (1,1,1),(3,3,3),(5,5,5),(15,15,15),(1,7,5),(3,21,15).\]
The possible ratios $(m/n, p/n)$ are hence $(1,1)$ and $(1/5, 7/5)$, and so \[a_4 - a_2 = a\left(\dfrac pn - \dfrac mn\right)\in\left\{0\cdot 1575, \frac65\cdot 1575\right\} = \boxed{\{0, 1890\}}.\] 

	\item One can define the greatest common divisor of two positive rational numbers as follows: for $a$, $b$, $c$, and $d$ positive integers with $\gcd(a,b)=\gcd(c,d)=1$, write \[\gcd\left(\dfrac ab,\dfrac cd\right) = \dfrac{\gcd(ad,bc)}{bd}.\] For all positive integers $K$, let $f(K)$ denote the number of ordered pairs of positive rational numbers $(m,n)$ wiht $m<1$ and $n<1$ such that \[\gcd(m,n)=\dfrac{1}{K}.\] What is $f(2017)-f(2016)$?

\proposed{David Altizio}

\solution First remark that the $\gcd$ condition can be dropped, since if $c$ and $d$ are scaled up by a factor of $k$, both $\gcd(ad,bc)$ and $bd$ are scaled up by $k$, and so their effects cancel out.

\par  Note that \[\dfrac ef\gcd\left(\dfrac ab,\dfrac cd\right) = \dfrac{e}{f}\cdot\dfrac{\gcd(ad,bc)}{bd} = \dfrac{\gcd(ead, ebc)}{bdf} = \gcd\left(\dfrac{ae}{bf},\dfrac{ce}{df}\right).\] Hence this definition of $\gcd$ is in fact multiplicative, and so it suffices to find pairs of rational numbers $m'$ and $n'$ such that $\gcd(m',n')=1$.

\par Write $m'=\tfrac{a'}{b'}$ and $n'=\tfrac{c'}{d'}$.  Then \[\gcd\left(\dfrac{a'}{b'},\dfrac{c'}{d'}\right) = 1\quad\iff\quad \gcd(a'd',b'c') = b'd'.\] Let $a'd' = Mb'd'$ and $b'c' = Nb'd'$ for some integers $M$ and $N$ with $\gcd(M,N) = 1$.  This simplifies to $\tfrac{a'}{b'} = M$ and $\tfrac{c'}{d'} = N$.  So in fact, $m'$ and $n'$ are actually relatively prime integers.

\par Hence $f(K)$ is equal to the number of pairs of positive integers $(M,N)$ with $1\leq M < K$ and $1\leq N < K$ such that $\gcd(M,N) = 1$.  This in turn means that $f(2017) - f(2016)$ equals the number of such pairs with either $M=2016$ or $N=2016$.  If $M = 2016$, then $N$ can be any one of the integers for which $\gcd(N, 2016) = 1$, of which there are $\varphi(2016)$ of them.  Similarly, $N=2016$ yields $\varphi(2016)$ more ordered pairs.  There is no possibility for overcounting, and so the final answer is \[2\varphi(2016) = \boxed{1152}.\]

	\item Find the largest positive integer $N$ satisfying the following properties:

\begin{itemize}

\item $N$ is divisible by $7$;

\item Swapping the $i^{\text{th}}$ and $j^{\text{th}}$ digits of $N$ (for any $i$ and $j$ with $i\neq j$) gives an integer which is \textit{not} divisible by $7$.

\end{itemize}

\proposed{David Altizio}

\solution Write \[N = \overline{a_ka_{k-1}\cdots a_1a_0} = \sum_{m=0}^k10^ma_m.\] Suppose digits $a_i$ and $a_j$ are swapped, whgere $0\leq i < j \leq k$, to form a new integer $N'$.  Then it is not hard to see that \[N - N' = \left(10^ja_j+10^ia_i\right)-\left(10^ja_i+10^ia_j\right) = \left(10^j-10^i\right)(a_j-a_i).\] The condition given in the problem statement is thus equivalent to this difference not being divisible by $7$ for all $i$ and $j$.

\par If $a_j-a_i$ is divisible by $7$, then $a_i\equiv a_j\pmod 7$.  This in turn means that all digits must have different residues modulo $7$.

\par If $10^j-10^i$ is divisible by $7$, then $10^{j-i}\equiv 1\pmod 7$.  Remark that $\operatorname{ord}_7(10) = 6$, meaning that it must be the case that $j-i\equiv 0\pmod 6$.  This means that any such $N$ must have at most $6$ digits; if this were not the case, then swapping $a_0$ and $a_6$ would produce a new integer divisible by $7$, thus violating the given conditions.

\par In all other cases, the difference will not be divisible by $7$.  Hence it suffices to find the largest integer $N$ with at most six digits such that $N\equiv 0\pmod 7$ and that each of the digits of $N$ has a different remainder when divided by $7$.  With this im mind, suppose $k=5$, and write \[N\equiv \sum_{m=0}^510^ma_m\equiv 5a_5+4a_4+6a_3+2a_2+3a_1+a_0\pmod 7.\] In the interest of being greedy, set $a_5=9$, $a_4=8$, and $a_3=7$; note that conveniently $987$ is divisible by $7$, so the search for possible $N$ is reduces to finding $a_0$, $a_1$, $a_2\in\{3,4,5,6\}$ such that \[2a_2+3a_1+a_0\equiv 0\pmod 7.\] Note that by the Rearrangement Inequality $2a_2+3a_1+a_0$ must be at least $2\cdot 4+3\cdot 3 + 5 = 22$ and at most $2\cdot 5 +3\cdot 6 + 4 = 32$.  Hence in fact it must be true that \[2a_2+3a_1+a_0 = 28.\] The only solutions to this under the given constraints is $(a_2,a_1,a_0)=(3,6,4)$ and $(a_2,a_1,a_0) = (5,4,6)$, so the largest $N$ must be $\boxed{987546}$. 

	\item The \textit{arithmetic derivative} $D(n)$ of a positive integer $n$ is defined via the following rules:

\begin{itemize}

\item $D(1) = 0$;

\item $D(p)=1$ for all primes $p$;

\item $D(ab)=D(a)b+aD(b)$ for all positive integers $a$ and $b$.

\end{itemize}

Find the sum of all positive integers $n$ below $1000$ satisfying $D(n)=n$.

\proposed{Varun Kambhampati}

\solution Let $N$ be a positive integer such that $D(N) = N$.  Recall that we can write \[N = p_1^{a_1}\cdots p_k^{a_k}\] for some sequence of primes $\{p_j\}_{j=1}^k$ and exponents $\{a_j\}_{j=1}^k$.  We now prove a lemma which explains how to compute arbitrary arithmetic derivatives.

\par\textbf{LEMMA: } We have \[D\left(p_1^{a_1}\cdots p_k^{a_k}\right) = p_1^{a_1}\cdots p_k^{a_k}\left(\dfrac{a_1}{p_1}+\cdots+\dfrac{a_k}{p_k}\right).\]

\begin{proof}

We first show that $D(p^j) = jp^{j-1}$ for $p$ a prime; this proves the claim in the case of $j=1$.  Fortunately, this is not hard. Write \[D(p^j) = p^{j-1}D(p) + pD(p^{j-1}) = p^{j-1}+pD(p^{j-1}).\] Now the claim follows from a simple induction argument.

\par To prove the lemma, we induct on $k$.  The base case of $k=1$ follows from the above paragraph.  Now assume the inductive hypothesis holds for some $k$, and write \begin{align*}D\left(p_1^{a_1}\cdots p_k^{a_k}\right) &= p_1^{a_1}\cdots p_{k-1}^{a_{k-1}}D\left(p_k^{a_k}\right) + p_k^{a_k}D\left(p_1^{a_1}\cdots p_{k-1}^{a_{k-1}}\right)\\&=p_1^{a_1}\cdots p_{k-1}^{a_{k-1}}\left(a_kp_k^{a_k-1}\right) + p_k^{a_k}\left(p_1^{a_1}\cdots p_{k-1}^{a_{k-1}}\right)\left(\dfrac{a_1}{p_1}+\cdots+\dfrac{a_{k-1}}{p_{k-1}}\right)\\&=p_1^{a_1}\cdots p_k^{a_k}\left(\dfrac{a_1}{p_1}+\cdots+\dfrac{a_k}{p_k}\right).\end{align*} Hence by induction we're done.

\end{proof}

Going back to the original problem, note that $D(N) = N$ implies that \[p_1^{a_1}\cdots p_k^{a_k}\left(\dfrac{a_1}{p_1}+\cdots+\dfrac{a_k}{p_k}\right) = p_1^{a_1}\cdots p_k^{a_k}\quad\implies\quad \dfrac{a_1}{p_1}+\cdots+\dfrac{a_k}{p_k} = 1.\] Multiplying both sides by $p_1\cdots p_k$ yields \[\dfrac{a_1p_1\cdots p_k}{p_1}+\cdots+\dfrac{a_kp_1\cdots p_k}{p_k} = p_1\cdots p_k.\] Now take both sides modulo $p_1$.  All but the first term goes away and so we are left with \[a_1p_2\cdots p_k \equiv 0\pmod{p_1}.\] Thus $p_1\mid a_1$.  However, since \[\dfrac{a_1}{p_1}+\cdots+\dfrac{a_k}{p_k} = 1,\] the ratio $\tfrac{a_1}{p_1}$ cannot exceed $1$.  Hence we in fact have equality, meaning that $p_1=a_1$ and $a_j=0$ for all $2\leq j\leq k$.  It follows that $N=p^p$ for prime $p$.  Since $5^5>1000$, the answer is simply $2^2+3^3=\boxed{31}$.

\item Let $N$ be the number of ordered triples $(a,b,c) \in \{1, \ldots, 2016\}^{3}$ such that $a^{2} + b^{2} + c^{2} \equiv 0 \pmod{2017}$. What are the last three digits of $N$?
	
	\proposed{Andrew Kwon}
	
	\solution We first claim that there are $2017^{2}$ solutions if we allow $a,b,c$ to equal 0. Letting $z$ be such that $z^{2} \equiv -1 \pmod{2017}$ (which we know exists because $2017$ is a prime congruent to $1\pmod 4$), the given congruence is equivalent to \[a^{2} \equiv (cz)^{2} - b^{2} = (cz - b)(cz+b) \pmod{2017}.\] If $cz - b \equiv 0$, then there is one choice for $a$ and $2017$ choices for $b$ from which $c$ is uniquely determined. Otherwise, we have $2016$ choices for the value of $cz-b$ and $2017$ choices for the value of $a$ from which the values $cz+b, b, c$ are determined. Thus, overall there are $2017^{2}$ triples $(a,b,c) \in \{0, \ldots, 2016\}^{3}$ satisfying the condition.

	\par We now use inclusion-exclusion to get the desired count. There are $2 \cdot 2017- 1$ triples where $a = 0$, as all choices for $b$ except $0$ yield two choices for $c$. The same is true for triples where $b = 0, c= 0$. This leads to a count of $2017^{2} - 6 \cdot 2017 + 3$, while the triple $(0,0,0)$ has been added once and removed thrice from the count, so we add 2 to get \[N = 2017^{2} - 6 \cdot 2017 + 5 = (2017 - 5)(2017 - 1),\] and the last three digits of $N$ are $2012 \cdot 2016 \equiv  \boxed{192} \pmod{1000}$.
	
\iffalse	
	As $2017$ is a prime which is 1 modulo 4, there is a value $z \in \mathbb{Z}/2017\mathbb{Z}$ such that $z^{2} \equiv -1 \pmod{2017}$, and so the desired congruence is equivalent to 
		\[
			\left( \frac{a}{cz} \right)^{2} +\left( \frac{b}{cz} \right)^{2} \equiv 1 \pmod{2017}.
		\]
		Thus it suffices to compute the number of pairs $(x,y) \in \{1,\ldots, 2016\}^{2}$ such that $x^{2} + y^{2} \equiv 1 \pmod{2017}$, where each $(x,y)$ corresponds to $2016$ possible $(a,b,c)$ tuples; there are 2016 choices for $c$, from which $a,b$ are determined. For ease of computation we take $M$ to be the number of solutions for $x,y \in \{0, \ldots, 2016\}$ instead. For the rest of our work we take $p = 2017$. By considering pairs $a,b \in \mathbb{F}_p$\footnote{Here we begin to use $\mathbb{F}_{p} = \mathbb{Z}/p\mathbb{Z}$ for ease of notation.} with $a+b=1$ and whether $a,b$ are quadratic residues we have
		\begin{align*}
			M &=  \sum_{\substack{a, b \in \mathbb{F}_{p}\\ a+b \equiv 1}}^{} \left( 1 + \left( \frac{a}{p} \right) \right) \left( 1 + \left( \frac{b}{p} \right) \right),
		\end{align*}
		where $(\frac{\cdot}{p})$ denotes the Legendre symbol. Expanding the above expression we find 
		\begin{align*}
			M &= p + \sum_{a \in \mathbb{F}_{p}}^{} \left( \frac{a}{p} \right) + \sum_{b \in \mathbb{F}_{p}}^{} \left( \frac{b}{p} \right) + \sum_{\substack{a,b \in \mathbb{F}_{p}\\ a + b \equiv 1}}^{} \left( \frac{a}{p} \right)\left( \frac{b}{p} \right),
		\end{align*}
		where $\sum_{a \in \mathbb{F}_{p}}^{} (\frac{a}{p}) = \sum_{b \in \mathbb{F}_{p}}^{} (\frac{b}{p}) = 0$. By convention $(\frac{0}{p}) = 0$, and so we need only consider $a \neq 1, b \neq 0$. Now,
		\begin{align*}
			\sum_{\substack{(a,b) \neq (1,0)\\ a + b \equiv 1}}^{} \left( \frac{a}{p} \right) \left( \frac{b}{p} \right) &= \sum_{\substack{(a,b) \neq (1,0)\\ a + b \equiv 1}}^{} \left( \frac{a}{p} \right) \left( \frac{b^{-1}}{p} \right)\\
			&= \sum_{1 \neq a \in \mathbb{F}_{p}}^{} \left( \frac{a/(1-a)}{p} \right),
		\end{align*}
		where $a/(1-a)$ takes all values in $\mathbb{F}_{p}$ except $-1$ as $a$ varies. With this in mind we deduce
		\[
			\left( \frac{-1}{p} \right) + \sum_{1 \neq a \in \mathbb{F}_{p}}^{} \left( \frac{a/(1-a)}{p} \right) = \sum_{c \in \mathbb{F}_{p}}^{} \left( \frac{c}{p} \right) = 0,
		\]
		and so 
		\[
			\sum_{1 \neq a \in \mathbb{F}_{p}}^{} \left( \frac{a/(1-a)}{p} \right) = - \left( \frac{-1}{p} \right).
		\]
		Finally,
		\[
			M = p - \left( \frac{-1}{p} \right),
		\]
		where $2017 \equiv 1 \pmod{4}$ and so $(\frac{-1}{p}) = 1$. Therefore we compute $M = p - 1 = 2016$. However, to count the number of pairs $(x,y) \in \{1, \ldots, 2016\}^{2}$ we subtract the 4 pairs where one of $x,y$ is equal to 0 to get a count of $p-5=2012$ pairs $(x,y) \in \{1, \ldots, 2016\}^{2}$. These yield $2016$ triples $(a,b,c)$ each, and so $N = 2012 \cdot 2016 \equiv \boxed{192} \pmod{1000}$.\\

		\textbf{Remark}. The expression 
		\[
			\sum_{a \in \mathbb{F}_{p}}^{} \left( \frac{a}{p} \right) \left( \frac{1-a}{p} \right)
		\]
		is a special instance of what are known as \textit{Jacobi sums} as a result of the fact that the Legendre symbol is a quadratic Dirichlet character.
\fi

\item Find the smallest prime $p$ for which there exist positive integers $a,b$ such that 
	\[
		a^{2} + p^{3} = b^{4}.
	\]

	\proposed{Andrew Kwon}

	\solution We rewrite the equation as $p^{3} = (b^{2} - a)(b^{2} + a)$, and as $b^{2} + a > b^{2} - a$ we have two cases.
	\begin{itemize}
		\item $b^{2} + a = p^{2}, b^{2} - a = p$: In this case, $2b^{2} = p(p+1)$, and noting that $p \neq 2$ we have $b^{2} = p (\tfrac{p+1}{2})$, from which we find $p | b$. However, then the right hand side must have at least two factors of $p$, while $p | \tfrac{p+1}{2}$ is impossible. Thus there are no solutions in this case.
		\item $b^{2} + a = p^{3}, b^{2} - a = 1$: In this case, $2b^{2} = (p+1)(p^{2} - p + 1)$, and we note again that $p \neq 2$. Now, $b^{2} = (\tfrac{p+1}{2})(p^{2} - p + 1)$, and we have 
			\[
				\gcd(\tfrac{p+1}{2}, p^{2} - p + 1) = \gcd(p+1, p^{2} - p + 1) = \gcd(p+1, 3).
			\]
			We split into further cases.
			\begin{itemize}
				\item $p \equiv 1 \pmod{3}$: As $\gcd(\tfrac{p+1}{2}, p^{2} - p + 1) = 1$, we must have that $\tfrac{p+1}{2}, p^{2} - p + 1$ are each perfect squares (since they are relatively prime and their product is a perfect square). Letting $n^{2} = \tfrac{p+1}{2}, m^{2} = p^{2} - p + 1$, we note that $n > 1$ and so 
					\[
						(2n^{2} - 2)^{2} < (2n^{2} - 1)^{2} - (2n^{2} - 1) + 1 = m^{2} < (2n^{2} - 1)^{2},
					\]
					which is impossible. Once again, we find no solutions.
				\item $p \equiv 2 \pmod{3}$: By an argument similar to before, we must have $\tfrac{p+1}{2}, p^{2} - p + 1$ are each 3 times a perfect square. Letting $3n^{2} = \tfrac{p+1}{2}, 3m^{2} = p^{2} - p + 1$ we find $p = 6n^{2} - 1, 3m^{2} = p^{2} - p + 1$. For $n=1$ we do have $p=5$, however there are no $m$ such that $3m^{2} = 21$. On the other hand, for $n=2$ we find $p = 23, p^{2} - p + 1 = 507 = 3 \cdot 169$, and so $\boxed{23}$ is the smallest valid value for $p$. Explicitly, $a = 6083, p = 23, b = 78$ is the complete solution to the original diophantine.
			\end{itemize}
	\end{itemize}
	
		\item For each positive integer $n$, define \[g(n) = \gcd\left\{0! n!, 1! (n-1)!, 2 (n-2)!, \ldots, k!(n-k)!, \ldots, n! 0!\right\}.\] Find the sum of all $n \leq 25$ for which $g(n) = g(n+1)$.

	\proposed{Cody Johnson and Andrew Kwon}

	\solution We claim $g(n) = \frac{(n+1)!}{\lcm(1, \ldots, n+1)}$, and it suffices to show 
	\[
		\nu_p(g(n)) = \nu_p((n+1)!) - \nu_p(\lcm(1, \ldots, n+1))
	\]
	for each prime $p$. Noting that $k!(n-k)! = n!/\binom{n}{k}$, we have
	\begin{align*}
		\nu_p(g(n)) &= \min_{1 \leq k \leq n} \left[\nu_p(n!) - \nu_p\left(\binom{n}{k}\right)\right]\\
		&= \nu_p(n!) - \max_{1 \leq k \leq n} \nu_p\left(\binom{n}{k}\right),
	\end{align*}
	and so 
	\begin{align*}
		\nu_p(g(n)) &= \nu_p((n+1)!) - \nu_p(\lcm(1, \ldots, n+1))\\
		\Leftrightarrow \nu_p(n!) - \max_{1 \leq k \leq n} \nu_p\left(\binom{n}{k}\right) &= \nu_p(n+1) + \nu_p(n!) - \max_{1 \leq j \leq n+1} \nu_p(j)\\
		\Leftrightarrow \max_{1 \leq j \leq n+1} \nu_p(j) &= \nu_p(n+1) + \max_{1 \leq k \leq n} \nu_p\left(\binom{n}{k}\right).
	\end{align*}
	Now, consider $\ell$ such that $p^{\ell} \leq n+1 < p^{\ell+1}$, so that $\max_{j} \nu_p(j) = \ell$. It suffices to show that $\nu_p(n+1) + \max_{k} \nu_p(\binom{n}{k}) = \ell$. Suppose for the sake of contradiction that $p^{\ell+1} | (n+1)\binom{n}{k}$ for some $k$. Note that 
	\[
		(n+1)\binom{n}{k} = (k+1) \binom{n+1}{k+1} = (n-k+1)\binom{n+1}{k},
	\]
	while evidently 
	\begin{align*}
		\nu_p\left(\binom{n}{k}\right) &= \sum_{s = 1}^{\ell} \left( \left\lfloor \frac{n}{p^{s}} \right\rfloor- \left\lfloor \frac{k}{p^{s}} \right\rfloor - \left\lfloor \frac{n-k}{p^{s}} \right\rfloor  \right)\\
		&\leq \ell,
	\end{align*}
	and similarly for $\binom{n+1}{k+1}, \binom{n+1}{k}$. Thus, we must have $p \mid n+1, k+1, n-k+1$, implying that $p \mid (n-k+1) + (k+1) - (n+1) = 1$, which is impossible. To finish, we note that $k = p^{\ell} - 1$ yields $p^{\ell} \mid (k+1) \binom{n+1}{k+1}$.

	\par We conclude that $g(n) = \frac{(n+1)!}{\lcm(1, \ldots, n+1)}$, and now claim that $g(n) = g(n+1) \Leftrightarrow n+2$ is prime. Supposing 
	\[
		\frac{(n+1)!}{\lcm(1, \ldots, n+1)} = \frac{(n+2)!}{\lcm(1, \ldots, n+2)},
	\]
	we have $n+2 = \frac{\lcm(m, n+2)}{m},$ where $m = \lcm(1, \ldots, n+1)$. Thus, $n+2$ is relatively prime to $1, \ldots, n+1$ and is prime. The other direction is clear.

	\par Thus, the desired $n \leq 25$ are $1, 3, 5, 9, 11, 15, 17, 21$ and their sum is $\boxed{82}$.

\end{enumerate}

\end{document}