\documentclass[10pt]{article}
\usepackage{amsmath, amssymb, amsthm}
%\usepackage{mathtools}
\usepackage[top=2cm, left = 2cm, right = 2cm, bottom = 3cm]{geometry}
\usepackage[pdftex]{graphicx}
\usepackage{asymptote}
\usepackage{fancyhdr}
\newcommand{\N}{\mathbb{N}}
\newcommand{\R}{\mathbb{R}}
\newcommand{\Z}{\mathbb{Z}}
\newcommand{\floor}[1]{\left\lfloor#1\right\rfloor}
\newcommand{\abs}[1]{\left\lvert#1\right\rvert}
\pagestyle{fancy}
\rhead{}
\chead{\includegraphics[scale=0.18]{CMIMC-header-2017.png}}
\lhead{}
\setlength{\headheight}{43pt}
\rfoot{}
\cfoot{}
\lfoot{}
\newcommand{\proposed}[1]
{
\vspace{5pt}
\noindent\textit{Proposed by #1}
}
\newcommand{\solution}
{
\vspace{5pt}
\noindent\textit{Solution.}\qquad
}
%\DeclarePairedDelimiter\abs{\lvert}{\rvert}

\begin{document}

\begin{center}
\huge\textbf{Algebra Solutions Packet}\normalsize

\vspace{3pt}
\end{center}

\begin{enumerate}

\item The residents of the local zoo are either rabbits or foxes. The ratio of foxes to rabbits in the zoo is $2:3$. After $10$ of the foxes move out of town and half the rabbits move to Rabbitretreat, the ratio of foxes to rabbits is $13:10$. How many animals are left in the zoo?

\proposed{Monica Pardeshi}

\solution Let $r$ be the number of rabbits and $f$ the number of foxes originally in the zoo.  Then $3f=2r$ and $\tfrac{13}2r = 10(f-10)$.  Solving for $f$, we have \[13r = \dfrac{39}2f = 20f - 200\quad\implies\quad f = 400.\] Substituting back in gives $r=600$, so the number of animals left is $(400-10)+\tfrac{600}2=\boxed{690}$.

\item For nonzero real numbers $x$ and $y$, define $x\circ y = \tfrac{xy}{x+y}$.  Compute \[2^1\circ \left(2^2\circ \left(2^3\circ\cdots\circ\left(2^{2016}\circ 2^{2017}\right)\right)\right).\]

\proposed{Patrick Lin}

\solution Rewrite $x\circ y$ as $\frac{1}{\frac1x+\frac1y}$.  Now note that for any $x,y,z$ with $xyz\geq 0$, \[x\circ (y\circ z) = \dfrac{1}{\frac1x+\frac1{\frac1{\frac1y+\frac1z}}} = \frac{1}{\frac1x+\frac1y+\frac1z}.\] Thus the entire expression becomes \[\dfrac{1}{\frac12+\frac1{2^2}+\cdots+\frac1{2^{2017}}} = \boxed{\dfrac{2^{2017}}{2^{2017}-1}}.\]

\item Suppose $P(x)$ is a quadratic polynomial with integer coefficients satisfying the identity \[P(P(x)) - P(x)^2 = x^2+x+2016\] for all real $x$.  What is $P(1)$?

\proposed{David Altizio}

\solution Let $P(x) = ax^2+bx+c$, so that $P(P(x)) = aP(x)^2+bP(x) + c$ and \[P(P(x)) - P(x)^2 = (a-1)P(x)^2+bP(x) + c.\] Since $\deg P = 2$, $\deg P^2 = 4$, so this expression will be a fourth-degree polynomial unless $a=1$.  Hence $P(x) = x^2+bx+c$, so the expression above simplifies to \[bP(x) + c = b(x^2+bx+c) + c = bx^2+b^2x+(bc+c).\] From here equating coefficients gives $b=1$ and $c=1008$, so $P(x)=x^2+x+1008$ and $P(1) = \boxed{1010}$.

\item It is well known that the the special mathematical constant $e$ can be written in the form $e = \tfrac{1}{0!}+\tfrac{1}{1!}+\tfrac{1}{2!}+\cdots$.  With this in mind, determine the value of \[\sum_{j=3}^\infty\dfrac{j}{\lfloor\frac j2\rfloor!}.\] Express your answer in terms of $e$.

\proposed{Joshua Siktar}

\solution Write \[\sum_{j=4}^\infty\dfrac{j}{\floor{\frac j2}!} = \sum_{k=2}^\infty\left(\dfrac{2k}{k!}+\dfrac{2k+1}{k!}\right) = \sum_{k=2}^\infty\dfrac{4}{(k-1)!}+\sum_{k=2}^\infty\dfrac{1}{k!}.\]  The first sum comes out to $4(e-\tfrac1{0!}) = 4e-4$, while the second come comes out to $e-\tfrac{1}{0!}-\tfrac{1}{1!} = e-2$.  Thus \[\sum_{j=4}^\infty\dfrac{j}{\floor{\frac j2}!} = (4e-4)+(e-2) = 5e-6.\] Adding back the $j=3$ term (which is $\tfrac{3}{1!}=3$) yields a final answer of $\boxed{5e-3}$.

\item The set $S$ of positive real numbers $x$ such that 
	\[
		\floor{\frac{2x}{5}} + \floor{\frac{3x}{5}} + 1 = \floor x
	\]
	can be written as $S = \bigcup_{j = 1}^{\infty} I_{j}$, where the $I_{i}$ are disjoint intervals of the form $[a_{i}, b_{i}) = \{x \, | \, a_i \leq x < b_i\}$ and $b_{i} \leq a_{i+1}$ for all $i \geq 1$. Find $\sum_{i=1}^{2017} (b_{i} - a_{i})$.

\proposed{Andrew Kwon}

\solution Say the disjoint intervals $I_{j}$ are \textit{funky}. Simple casework yields $[1,\tfrac{5}{3}), [2, \tfrac{5}{2}), [3, \tfrac{10}{3}), [4,5)$ as the only funky intervals in $[0,5)$.\footnote{A simple way to perform this casework systematically is as follows: define the function $f:\R\to\Z$ via \[f(x) = \floor{\frac{2x}{5}} + \floor{\frac{3x}{5}} - \floor x.\] Note that this quantity increases by $1$ at every multiple of $\tfrac52$ and $\tfrac53$ and decreases by $1$ at every integer $x$.  Thus, one can count how many such increases and decreases are made and examine the places at which the function equals one.} Furthermore, we note that 
\[
	\floor{\frac{2(x+5)}{5}} + \floor{\frac{3(x+5)}{5}} + 1 = \floor{\frac{2x}{5}} + \floor{\frac{3x}{5}} + 6,
\]
and so $x$ is in a funky interval $\Leftrightarrow x+5$ is in a funky interval. Therefore, all funky intervals are translations of the funky intervals found in $[0,5)$. It is easy to see then that $\sum_{i=1}^{2016} (b_{i} - a_{i}) = \tfrac{5}{2} \cdot \tfrac{2016}{4} = 1260$, and $b_{2017} - a_{2017} = \frac{2}{3}$. The final answer is $\boxed{\tfrac{3782}{3}}$.

\item Suppose $P$ is a quintic polynomial with real coefficients with $P(0)=2$ and $P(1)=3$ such that $|z|=1$ whenever $z$ is a complex number satisfying $P(z) = 0$.  What is the smallest possible value of $P(2)$ over all such polynomials $P$?

\proposed{David Altizio}

\solution Note that complex roots of $P$ must come in conjugate pairs.  Since the degree of $P$ is odd, $P$ must have one real root, and by the $|z|=1$ condition this root must be either $1$ or $-1$.  However, $P(1) \neq 0$, so $-1$ must be said root.  Now let $\alpha$, $\bar\alpha$, $\beta$, and $\bar\beta$ be the remaining four roots.  (This implicitly covers the real case as well, since it's impossible for one real root of $P$ to be $1$ and the other to be $-1$.)  This implies that \begin{align*}P(z) &= C(z+1)(z-\alpha)(z-\bar\alpha)(z-\beta)(z-\bar\beta)\\&= C(z+1)(z^2-(\alpha+\bar\alpha)z + \alpha\bar\alpha)(z^2-(\beta+\bar\beta)z + \beta\bar\beta)\\&=C(z+1)(z^2-2\Re(\alpha)z+1)(z^2-2\Re(\beta)z+1),\end{align*} where we use $\alpha\bar\alpha = |\alpha|^2 = 1$ and similar in the last step.  For ease of typesetting, let $a=2\Re(\alpha)$ and $b=2\Re(\beta)$, so that $P(z) = C(z+1)(z^2-az+1)(z^2-bz+1)$ for $|a|,|b|\leq 2$.  Plugging in $z=0$ gives $C=2$, while plugging in $z=1$ yields \[3 = 2\cdot 2(2-a)(2-b)\quad\implies\quad (2-a)(2-b)=\frac34.\] It thus suffices to minimize \[P(2) = 2\cdot 3(2^2-2a+1)(2^2-2b+1) = 6(5-2a)(5-2b)\] subject to the constraints given above.

\par Once again, for ease of typesetting set $p=2-a$ and $q=2-b$.  Then $pq=\tfrac34$ and \[(5-2a)(5-2b)=(2p+1)(2q+1) = 4pq+2(p+q)+1=4+2(p+q).\] This means that we must minimize $p+q$.  Note that since $|a|\leq 2$ and $|b|\leq 2$, $p$ and $q$ are both nonnegative, so we may apply the AM-GM inequality to obtain $p+q\geq 2\sqrt{pq} = \sqrt 3$.  Thus the smallest possible value of $P(2)$ is \[6(5-2a)(5-2b) = 6\cdot[4+2(p+q)] =\boxed{24+12\sqrt 3}.\] Note that equality is achieved via \[P(z) = 2(z+1)\left(z^2-\left(2-\frac{\sqrt 3}2\right)z+1\right)^2.\]

\item Let $a$, $b$, and $c$ be complex numbers satisfying the system of equations \begin{align*}\dfrac{a}{b+c}+\dfrac{b}{c+a}+\dfrac{c}{a+b}&=9,\\\dfrac{a^2}{b+c}+\dfrac{b^2}{c+a}+\dfrac{c^2}{a+b}&=32,\\\dfrac{a^3}{b+c}+\dfrac{b^3}{c+a}+\dfrac{c^3}{a+b}&=122.\end{align*} Find $abc$.

\proposed{David Altizio}

\solution Let \[E_r = \dfrac{a^r}{b+c}+\dfrac{b^r}{c+a}+\dfrac{c^r}{a+b}\] for all nonnegative integers $r$.  Note that \begin{align*}E_{r+1} + (a^r+b^r+c^r) &= \dfrac{a^{r+1}}{b+c}+\dfrac{b^{r+1}}{c+a}+\dfrac{c^{r+1}}{a+b} + (a^r+b^r+c^r) \\&= \left(\dfrac{a^{r+1}}{b+c} + a^r\right) + \left(\dfrac{b^{r+1}}{c+a}+b^r\right)+\left(\dfrac{c^{r+1}}{a+b} + c^r\right)\\&=\dfrac{a^{r+1}+a^rb+a^rc}{b+c}+\dfrac{b^{r+1}+b^rc+b^ra}{c+a}+\dfrac{c^{r+1}+c^ra+c^rb}{a+b}\\&=(a+b+c)\left(\dfrac{a^r}{b+c}+\dfrac{b^r}{c+a}+\dfrac{c^r}{a+b}\right) = (a+b+c)E_r.\end{align*} This is this identity that will be the workhorse for our solution.

\par Note that plugging in $r=1$ gives $32 + (a+b+c) = 9(a+b+c)$, or $a+b+c=4$.  Similarly, note that the $r=2$ case gives $122 + (a^2+b^2+c^2) = 32(a+b+c)=128\implies a^2+b^2+c^2=6$.  Next, the $r=0$ case yields $9+3 =4(\tfrac1{a+b}+\tfrac1{b+c}+\tfrac{1}{c+a})$, and so $\tfrac1{a+b}+\tfrac1{b+c}+\tfrac{1}{c+a}=3$.  Now write \begin{align*}\dfrac{1}{4-a}+\dfrac{1}{4-b}+\dfrac{1}{4-c} &= 3 \\ \implies (4-a)(4-b)+(4-a)(4-c) + (4-b)(4-c) &= 3(4-a)(4-b)(4-c)\\\implies 48 - 8(a+b+c) + (ab+bc+ca) &= 3(64 - 16(a+b+c) + 4(ab+bc+ca) - abc)\\&= 12(ab+bc+ca) - 3abc\\\implies 11(ab+bc+ca) - 16 &= 3abc.\end{align*} Finally, recall that $a+b+c=4$ and $a^2+b^2+c^2 = 6$ implies $ab+bc+ca = 5$, so \[11(5) - 16 = 39 = 3abc\quad\implies\quad abc=\boxed{13}.\]

\item Suppose $a_1$, $a_2$, $\ldots$, $a_{10}$ are nonnegative integers such that \[\sum_{k=1}^{10}a_k=15\qquad\text{and}\qquad \sum_{k=1}^{10}ka_k = 80.\] Let $M$ and $m$ denote the maximum and minimum respectively of $\sum_{k=1}^{10}k^2a_k$.  Compute $M-m$.

\proposed{David Altizio}

\solution The key to this problem is the following trick: let $m$ and $k$ be integers between $1$ and $10$ inclusive.  Suppose $(a_{m-1},a_m,a_k,a_{k+1})$ are four elements of a tuple satisfying the given conditions.  Replace this tuple with \[(a_{m-1}-1,a_m+1,a_k+1,a_{k+1}-1).\] It's easy to see that both equalities are still satisfied, but now \begin{align*}&(m-1)^2(a_{m-1}-1) + m^2(a_m+1) + k^2(a_k+1) + (k+1)^2(a_{k+1}-1)\\&=V + m^2-(m-1)^2 + k^2 - (k+1)^2\\&=V + 2(m-k) - 2,\end{align*} where here $V = (m-1)^2a_{m-1}+m^2a_m+k^2a_k+(k+1)^2a_{k+1}$.  Hence, as long as $m\leq k$, performing such an operation will decrease the value of $\sum_{k=1}^{10}k^2a_k$.  Conversely, if $m-k\geq 1$, such an operation will increase the value of the requested quantity.

\par First we compute $m$.  It is easy to see the minimum value of our expression comes when there exists a $j$ such that only $a_j$ and $a_{j+1}$ are nonzero; otherwise, we could apply this operation with $m-1$ the smallest index $k$ such that $a_k>0$ and $n+1$ the largest such $k$ to decrease $\sum_{k=1}^{10}k^2a_k$ even further.  This $j$ must satisfy \[a_j+a_{j+1} = 15\qquad\text{and}\qquad ja_j+(j+1)a_{j+1} = 80.\] Note that the second equation becomes \[j(a_j+a_{j+1}) + a_{j+1} = 15j+a_{j+1} = 80.\] Now remark that by integer bounding the only possible value of $j$ is $j=5$, which gives $a_{j+1} = 5$.  Hence $a_5 = 10$ and $a_6 = 5$, so \[m=5^2\cdot 10 + 6^2\cdot 5 = 430.\]

\par Computing $M$ is similar, but the required conditions are a bit trickier.  First remark that the system of equations \[\begin{cases}a_1+a_{10} &= 15,\\a_1 + 10a_{10} &= 80\end{cases}\] has unique solution $(a_1,a_{10}) = (\tfrac{70}9,\tfrac{65}9)$; these are not integers, and as such it is impossible for only $a_1$ and $a_{10}$ to be nonzero.  With this in mind, we claim that the sum is minimized under the condition that \[a_2+a_3+\cdots +a_9 = 1;\] in other words, exactly one of these numbers is $1$ and the rest are zeros.  To see this, suppose the contrary.  Write each of $a_2$ through $a_9$ as a sum of $1$s (so for example, $2=1+1$).  Pick two of these ones, supposing they come from $a_j$ and $a_k$ with $j\leq k$.  Now by repeatedly applying the operation \[(0,1,\ldots, 1,0)\mapsto (1,0,\ldots, 0,1),\] we can force at least one of these ones out toward the edges to either $a_1$ or $a_{10}$.  This means that the quantity $a_2+\cdots +a_9$ decreases by at least one.  The claim follows by an inductive argument on this quantity.

\par As such, in order for the maximum to be achieved, we need \[\begin{cases}a_1+a_{10}&=14,\\a_1+10a_{10}&=80-k\end{cases}\] for some integer $2\leq k\leq 9$.  Subtracting the equations and taking mod $9$ yields \[0\equiv 9a_{10}\equiv 66-k\equiv 3-k\pmod 9\quad\implies\quad k=3.\] Now solving the resulting system gives $(a_1,a_{10}) = (7,7)$, so \[M = 1^2\cdot 7 + 3^2\cdot 1 + 10^2\cdot 7 = 716\] and the requested answer is $716 - 430 = \boxed{286}$.

\item Define a sequence $\{a_{n}\}_{n=1}^{\infty}$ via $a_{1} = 1$ and $a_{n+1} = a_{n} + \lfloor \sqrt{a_{n}} \rfloor$ for all $n \geq 1$. What is the smallest $N$ such that $a_{N} > 2017$?

	\proposed{Andrew Kwon}

	\solution We first claim that all powers of 4 appear in this sequence, and that these are the only perfect squares in this sequence. Evidently $a_{1} = 1, a_{4} = 4$, and so the claim is not false yet.\\

	In general, for $k \geq 2$ suppose $a_{k} = n^{2} + r$ with $1 \leq r \leq n$.\footnote{We need not seriously consider the case $n + 1 \leq r \leq 2n$, as $a_{k+1} = n^{2} + n + r$, and when $1 \leq r \leq n$ we have $n+ 1 \leq n + r \leq 2n$.} Then, $a_{k+2} = n^{2} + 2n + r = (n+1)^{2} + (r-1)$, and inductively we find $a_{k + 2r} = (n+r)^{2}$. Furthermore, none of the terms between $a_{k}, a_{k+2r}$ are perfect squares. In particular, if $a_{k-1} = n^{2}$, then $a_{k} = n^{2} + n$ and $a_{k + 2n} = 4n^{2}$. As we have verified that the first perfect squares in our sequence are 1 and 4, the only perfect squares in our sequence are powers of $4$.\\

	It is not hard to see that $\lfloor \sqrt{a_{n}} \rfloor$ will attain all positive integer values, but we claim that it will attain powers of 2 three times, and all other values twice. Indeed, if $n^{2} + n \leq a_{k} \leq n^{2} + 2n$ for some $n$, then we must have $n^{2} \leq a_{k-1} \leq n^{2} + n$, and so $a_{k-1}, a_{k} \in [n^{2}, (n+1)^{2})$. This corresponds to $\lfloor \sqrt{a_{k-1}}\rfloor, \lfloor \sqrt{a_{k}}\rfloor = n$. The only way for three terms $a_{k-1}, a_{k}, a_{k+1}$ to be in the interval $[n^{2}, (n+1)^{2})$ is if $a_{k-1} = n^{2}, a_{k} = n^{2} + n$, and $a_{k+1} = n^{2} + 2n$. This is precisely when $\lfloor \sqrt{a_{k-1}} \rfloor, \lfloor \sqrt{a_{k}}\rfloor, \lfloor\sqrt{a_{k+1}}\rfloor$ are powers of 2.\\

	Now we proceed by consideration of adding consecutive differences. We consider 
	\[
		a_{N} = 2(1 + 2 + \ldots + k) + (1 + 2 + \ldots + 2^{\ell-1}) > 2017
	\]
	or
	\[
		a_{N} = 2(1 + 2 + \ldots + k) + (1 + 2 + \ldots + 2^{\ell}) > 2017,
	\]
	where $\ell$ is the unique integer such that $2^{\ell} \leq k < 2^{\ell+1}$ and we add $1 + \ldots + 2^{\ell-1}$ or $1 + \ldots + 2^{\ell}$ because those differences appear three times rather than twice, but we do not yet know whether the third contribution of $2^{\ell}$ is necessary or not. Now the above expressions are equivalent to $k^{2} + k + 2^{\ell}$ and $k^{2} + k + 2^{\ell + 1}$. As $43 \cdot 44 = 1892, 44\cdot 45 = 1980$ we find $k = 44$ suffices to guarantee $a_{N} = 2044 > 2017$ when we include $2^{\ell} = 32$. To determine the value of $N$, we use the fact that we have added $2 \cdot 44 + 6$ consecutive differences, and so cumulatively we have calculated the $95^{\text{th}}$ term of the sequence, and $N = \boxed{95}$ is minimal.\\



\item Let $c$ denote the largest possible real number such that there exists a nonconstant polynomial $P$ with \[P(z^2)=P(z-c)P(z+c)\] for all $z$.  Compute the sum of all values of $P(\tfrac13)$ over all nonconstant polynomials $P$ satisfying the above constraint for this $c$.

\proposed{David Altizio}

\solution We claim that $c=\tfrac12$.

\par First note that if $\alpha$ is a root of $P$, then plugging in $z=\alpha+c$ yields \[P((\alpha+c)^2) = P(\alpha)P(\alpha+2c) = 0,\] so that $(\alpha+c)^2$ is a root of $P$ as well.  Similarly, $(\alpha-c)^2$ must also be a root of $P$.

Now suppose $c > \frac12$, and let $z$ be a possible root of $P$.  Define a sequence of complex numbers $\{z_k\}_{k=0}^\infty$ such that $z_0=z$ and such that $z_{k+1}$ is either equal to $(z_k+c)^2$ or $(z_k-c)^2$.  I claim it is always possible to choose a sequence with the property that the sequence $\{|z_k|\}_{k=0}^\infty$ is strictly increasing.  To see this, recall by the Parallelogram Law, \[|z-c|^2+|z+c|^2 = 2(|z|^2+c^2).\] It thus follows that one of $|z-c|^2$ and $|z+c|^2$ must be at least $|z|^2+c^2$ (else the entire sum would be too small), so we can choose $z_{k+1}$ such that $|z_{k+1}|\geq |z_k|^2+c^2$.  But note that \[|z|^2+c^2>|z|\quad\iff\quad \left(|z|-\frac12\right)^2 + c^2 > \frac14,\] which is always true for $c>\tfrac12$.  Thus $|z_{k+1}| > |z_k|$, as desired.  It follows that $\{z_k\}_{k=0}^\infty$ is an infinite sequence of roots of $P$, which is a contradiction.

\par It suffices to classify all polynomials satisfying the equation when $c=\tfrac12$.  To do this, remark that there are two equality cases in the above analysis.  The first occurs in the choice of $z_{k+1}$; equality here occurs when $|z-c|^2 = |z+c|^2$, or when $z$ is purely imaginary.  The second equality case occurs in completing the square.  For $c=\tfrac12$, we need $(|z|-\tfrac12)^2=0$, i.e. $|z|=\tfrac12$.  It follows that $\tfrac12i$ and $-\tfrac12i$ are the only possible roots of $P$, and furthermore it is easy to see that these roots must occur with equal multiplicity.  Indeed, taking $P(z) = z^2+\tfrac14$, we see that \begin{align*}P\left(z-\frac12\right)P\left(z+\frac12\right)&=\left(\left(z-\frac12\right)^2+\frac14\right)\left(\left(z+\frac12\right)^2+\frac14\right) \\&= \left(z^2-z+\frac12\right)\left(z^2+z+\frac12\right)\\&=\left(z^2+\frac12\right)^2-z^2 = z^4+\frac14 = P(z^2).\end{align*}  Hence $P(z) = (z^2+\tfrac14)^n$ for some integer $n\geq1$, and it follows that the sum of all possible values of $P(\tfrac13)$ is \[\sum_{n\geq 1}\left(\dfrac19+\frac14\right)^n = \sum_{n\geq 1}\left(\frac{13}{36}\right)^n = \boxed{\frac{13}{23}}.\]

\end{enumerate}

\end{document}
